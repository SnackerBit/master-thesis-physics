Quantum many-body systems exhibit a variety of novel and exotic emergent behaviours arising from the interaction of many local degrees of freedom. Well-known examples of such phenomena are the fractional quantum Hall effect \cite{cite:two_dimensional_magnetotransport_in_the_extreme_quantum_limit, cite:anomalous_quantum_hall_effect_an_incompressible_quantum_fluid_with_fractionally_charged_excitations, cite:the_fractional_QHE} and other spin liquids with fractionalized excitations \cite{cite:fractionalized_excitations_in_the_spin_liquid_state_of_a_kagome_lattice_AFM}, topological phases of matter \cite{cite:colloqium_zoo_of_quantum_topological_phases_of_matter}, and high-temperature superconductivity \cite{cite:possible_high_T_c_superconductivity_in_the_Ba_La_Cu_O_system, cite:doping_a_mott_insulator_physics_of_high_temperature_superconductivity}. The main challenge in the study of quantum many-body systems is posed by the fact that the dimension of the Hilbert space grows exponentially with the system size. For example, exact diagonalization of spin-$1/2$ models is limited to systems of a few tens of spins. There has been a lot of effort to overcome this problem by introducing numerical algorithms that approximately simulate quantum many-body systems, with the two most prominent approaches being Quantum Monte Carlo (QMC) \cite{cite:quantum_monte_carlo_simulation_method_for_spin_systems, cite:computational_studies_of_quantum_spin_systems} and tensor network algorithms. While QMC algorithms have had a lot of success, the infamous sign problem prevents the method from simulating many interesting models, for example frustrated spin systems \cite{cite:computational_studies_of_quantum_spin_systems, cite:sign_problem_in_numerical_simulation_of_many_electron_systems}. Tensor network methods have had great success in the simulation of one-dimensional systems. Especially the Density Matrix Renormalization Group (DMRG) algorithm \cite{cite:density_matrix_formulation_for_quantum_renormalization_groups}, which was later understood as a variational method over the class of Matrix Product States (MPS) \cite{cite:equivalence_of_the_variational_mps_and_the_dmrg_applied_to_spin_chains, cite:DMRG_in_the_age_of_MPS, cite:practical_introduction_MPS_and_PEPS}, has developed to be the de-facto standard for the computation of ground-state properties of gapped Hamiltonians. Such ground states exhibit a characteristic area-law entanglement structure \cite{cite:area_law_1D_proof} and the success of DMRG is due to the remarkable ability of MPS to capture this entanglement structure efficiently \cite{cite:area_laws_review, cite:mps_represent_ground_states_faithfully}. Another important problem is the real and imaginary time evolution of quantum states, which can for example be performed y using the the Time Evolving Block Decimation (TEBD) \cite{cite:efficient_simulation_of_1D_quantum_many_body_systems} algorithm or the Time Dependent Variational Principle (TDVP) \cite{cite:time_dependent_variational_principle_for_quantum_lattices}.
The natural generalization of MPS to two and higher dimensions is given in the form of Projected Entangled Pair States (PEPS) \cite{cite:practical_introduction_MPS_and_PEPS, cite:renormalization_algorithms_for_qmb_systems_in_two_and_higher_dimensions}, which are able to efficiently represent area-law states \cite{cite:practical_introduction_MPS_and_PEPS}. However, algorithms for ground state search and time evolution of PEPS have a computational cost scaling with high powers of the bond dimension $D$. For example, the cost of a full time evolution update scales as $\mathcal{O}(D^{10})$ and the cost of variational energy minimization scales as $\mathcal{O}(D^{12})$ \cite{cite:unifying_PEPS_contractions, cite:algorithms_for_finite_PEPS}. Additionally, the energy-minimization algorithm that is obtained by generalizing DMRG to PEPS requires solving a generalized eigenvalue problem, which is numerically ill-conditioned \cite{cite:algorithms_for_finite_PEPS}. \par
Recently, the new class of isometric Tensor Product States (isoTPS) have been introduced \cite{cite:isometric_tensor_network_states_in_two_dimensions, cite:conversion_of_PEPS_into_a_canonical_form, cite:DMRG_approach_to_optimizing_2D_tensor_networks}, generalizing the canonical form of MPS to two and higher dimensions by enforcing isometry constraints. This allows for the efficient computation of local expectation values and can reduce the computational cost of algorithms compared to PEPS. For example, the cost of real and imaginary time evolution is reduced from $\mathcal{O}(D^{10})$ to $\mathcal{O}(D^7)$ \cite{cite:isometric_tensor_network_states_in_two_dimensions}. While the generalization of DMRG to isoTPS still has a computational complexity of $\mathcal{O}(D^{12})$, the generalized eigenvalue problem is reduced to a standard eigenvalue problem because of the canonical form, improving the stability of the algorithm \cite{cite:efficient_simulation_of_dynamics_in_two_dimensional_quantum_spin_systems}. The downside to this approach is that not all quantum states that can be represented by PEPS can also be represented by isoTPS. It is thus an interesting question to ask which kinds of states and, more generally, which kinds of quantum phases can still be efficiently written as an isoTPS. The expressional power of isoTPS has been studied in \cite{cite:isometric_tensor_network_representation_of_string_net_liquids}, where it was found that isoTPS with finite bond dimension can exactly represent ground state wavefunctions of string-net liquid models, showing that long-range entanglement does not form an obstruction for isoTPS representations and suggesting that the ground states of gapped Hamiltonians with gappable edges can be efficiently represented as an isoTPS. There have also been works discussing the computational complexity of isoTNS contractions \cite{cite:computational_complexity_of_isometric_tensor_network_states} and relating isoTNS to quantum circuits \cite{cite:sequential_generation_of_projected_entangled_pair_states, cite:quantum_circuits_for_2D_isometric_tensor_networks}. In \cite{cite:topological_quantum_phase_transitions_in_2D_isometric_tensor_networks}, topological phase transitions were studied with isoTPS, showing that isoTPS can represent some critical states with power-law correlations. IsoTPS were also extended to fermionic systems \cite{cite:fermionic_isometric_tensor_network_states}, to two dimensional strips of infinite length \cite{cite:two_dimensional_isometric_tensor_networks_on_infinite_strip}, and to three dimensional cubic lattices \cite{cite:three_dimensional_isometric_tensor_networks}. They have also been used to compute properties of two dimensional thermal states \cite{cite:isometric_tensor_network_representation_of_2D_thermal_states}. \par
While algorithms on isoTPS have shown first promising results, there are still open questions. For example, the best way of defining the canonical form is not yet agreed upon. Different canonical forms could in principle lead to reduced errors, more stable algoritms, and reduced computational cost. In this work we propose a new variant of isoTPS which we call diagonal isometric Tensor Product States (disoTPS). The ansatz differs from isoTPS by rotating the lattice by $45^\circ$ and introducing auxillary tensors that do not have a physical degree of freedom. While we mainly discuss the implementation on the square lattice, disoTPS are easily generalizable to other lattice types.
\par
The thesis is structured as follows. First, an introduction to tensor networks, MPS and isoTPS is provided in chapter \ref{chap:tensors_and_tensor_networks}. Next, the new class of disoTPS is introduced on the square lattice in chapter \ref{chap:disoTPS} and a TEBD algorithm for real and imaginary time evolution is formulated. In chapter \ref{chap:toric_code} it is shown that disoTPS are able to represent the ground state of the Toric Code model, a quantum spin model with $\mathbb{Z}_2$ topological order. The method is then further benchmarked on the Transverse Field Ising (TFI) model in chapter \ref{chap:TFI}, showcasing the ability of disoTPS to find ground states and perform real time evolution. Additionally, disoTPS are generalized to the honeycomb lattice. Last, a summary and outlook are given in chapter \ref{chap:summary}.
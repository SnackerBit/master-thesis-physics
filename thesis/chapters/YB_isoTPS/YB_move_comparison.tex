\begin{figure}
	\centering
	\begin{tikzpicture}[scale=1, trim axis left, trim axis right]
		\begin{axis}[xlabel={walltime [s]}, ylabel={$\lVert|\Psi\rangle-|\Psi^\prime\rangle\rVert$}, grid=both, grid style={gray!20}, every axis plot/.append style={very thick}, scale only axis, height=\singleFigureHeight, width=\singleFigureWidth, xmode=log, legend pos=north east, legend style={nodes={scale=\legendscale, transform shape}}]
			%
			\addplot[color = blue]
			table[x=t, y=trunc_error, col sep=space]{figures/plots/disoTPS/data/comparing_tripartite_decompositions_iterate_polar.txt};
			\addlegendentry{EV trunc}
			%
			\addplot[color = red]
			table[x=t, y=trunc_error, col sep=space]{figures/plots/disoTPS/data/comparing_tripartite_decompositions_renyi_2_EV.txt};
			\addlegendentry{EV Rényi-$2$}
			%
			\addplot[color = 5orange4]
			table[x=t, y=trunc_error, col sep=space]{figures/plots/disoTPS/data/comparing_tripartite_decompositions_renyi_0.5_approx_trm.txt};
			\addlegendentry{appr.\,CG\,Rényi-$0.5$}
			%	
			\addplot[color = 5green4]
			table[x=t, y=trunc_error, col sep=space]{figures/plots/disoTPS/data/comparing_tripartite_decompositions_trunc_approx_trm.txt};
			\addlegendentry{appr. TRM trunc}
			%
		\end{axis}
	\end{tikzpicture}
	\caption{\todo{caption! + choose better colors in plot + better legend placement!}}
	\label{fig:yb_move_comparison}
\end{figure}
In the following, we qualitatively compare the two discussed algorithms for the YB move. We again select the YB move environment $\{W_1,W_2, T\}$ that we used for discussing the individual convergence of the iterative YB move in Section \ref{sec:YB_move_iterative_local_optimization} and the disentanglers in Section \ref{sec:YB_move_svd_disentangle}. We compare five algorithms for performing the YB move: A simple SVD splitting without a disentangling algorithm, using a naive guess for the disentangling unitary that is computed by performing an SVD as in \cite{cite:isometric_tensor_network_states_in_two_dimensions, cite:efficient_simulation_of_dynamics_in_two_dimensional_quantum_spin_systems} (\textbf{SVD + init}), The iterative Evenbly-Vidal-style optimization of the truncation error (\textbf{EV trunc}) from Section \ref{sec:YB_move_iterative_local_optimization}, the Evenbly-Vidal-style disentangler minimizing the Rényi-2-entropy (\textbf{EV Rényi-2.0}), and the approximate TRM minimizing the truncation error (\textbf{approx TRM trunc}) and the Rényi-0.5-entropy (\textbf{approx TRM Rényi-0.5}). We plot the error of the YB move $\lVert\ket{\Psi}-\ket{\Psi^\prime}\rVert$ against the walltime in seconds in Figure \figref{fig:yb_move_comparison}. Note that this is only a qualitative comparison, since the environment was selected arbitrarily. For a more quantitative comparison, see Chapter \ref{chap:TFI}. We observe that the algorithms that directly minimize the truncation error also are able to achieve a lower YB error than the algorithms minimizing the Rényi-entropy. However, as we will see in Chapter \ref{chap:TFI}, minimizing the Rényi-entropy actually leads to a lower global error when performing a full shifting of the orthogonality hypersurface. We further observe the slow convergence of the iterative Evenbly-Vidal style algorithm, suggesting that the disentangling algorithm with Riemannian optimization of the truncation error will provide better results. Even though the computational complexity of \textbf{EV Rényi-2.0} scales as $\mathcal{O}(D^9)$, its quick convergence behaviour makes it faster than all other algorithms. Because of this, we use the algorithm as initialization for the disentanglers using Riemannian optimization. Note that in principle the approximate SVD discussed in Section \ref{sec:YB_move_svd_disentangle} could be used to decrease the computational cost of \textbf{EV Rényi-2.0} to $\mathcal{O}(N_\text{iter}(D^8 + N_\text{svd} D^7))$, which was not necessary for the bond dimensions we used.
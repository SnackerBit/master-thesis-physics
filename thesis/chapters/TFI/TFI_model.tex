The Transverse Field Ising (TFI) model is a well-studied spin lattice model that is described by the Hamiltonian
\begin{equation}
	\label{eq:TFI_Hamiltonian}
	\hat{H}_\text{TFI} = -J\sum_{\langle i,j\rangle} \hat{\sigma}^x_i \hat{\sigma}^x_j - g\sum_{i} \hat{\sigma}^z_i,
\end{equation}
where $\langle i,j\rangle$ denotes pairs of nearest-neighbor spin-1/2 particles and $\hat{\sigma}^x_i, \hat{\sigma}^z_i$ are the Pauli matrices. Here we will only discuss the TFI model on a two-dimensional lattice, which can be mapped to a classical Ising model on a three-dimensional lattice \cite{cite:from_d_dimensional_quantum_to_dp1_dimensional_classical}. We will further restrict ourselves to the ferromagnetic case $J > 0$ and to zero temperature. Since the behaviour of the model is then only controlled by the ratio $g/J$, we set $J = 1$ and only vary $g$. In the limit of vanishing transverse field $g \rightarrow 0$ the model reduces to a classical 2D Ising model. The ground state is degenerate with all spins pointing either up or down in the $S^x$ direction. The associated phase is the classically disordered ferromagnetic phase \cite{cite:critical_behavior_of_the_two_dim_ising_model_in_transverse_field, cite:quantum_ising_phases_and_transitions_in_transverse_ising_models}. Taking the other limit, $g \gg J$, reduces the model to non-interacting spins in an external field. The ground state is unique with all spins pointing in $S^z$ direction. The corresponding phase is the paramagnetic phase. There exists a quantum phase transition at a critical transverse field $g = g_\text{C}$ from the ferromagnetic to the paramagnetic phase. Blöte and Deng computed the value of the critical field for the TFI model on the square lattice as $g \approx 3.04438$ using Quantum Monte Carlo methods \cite{cite:cluster_monte_carlo_simulation_of_TFI}. \par
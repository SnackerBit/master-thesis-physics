%\usetikzlibrary{backgrounds} % DEBUG
%background rectangle/.style={fill=olive!45}, show background rectangle
\begin{figure}
	\centering
	\begin{minipage}{1.0\textwidth}
		\centering
		\begin{tikzpicture}[scale=1, trim axis left, trim axis right]
			\begin{axis}[xlabel=$t$, ylabel={$\langle\hat{\sigma}_z\rangle$}, grid=both, grid style={gray!20}, every axis plot/.append style={very thick}, scale only axis, height=\singleFigureHeight, width=\singleFigureWidth, legend style={at={(1.2715,0.9)}, anchor=north west, font=\small, nodes={scale=\legendscale, transform shape}, label={[font=\small]above:{MPS DMRG}}}, legend columns=1, legend cell align={left}, xmin=-0.05, xmax=1.05, ymin=0.78, ymax=1.02]
				%
				\addplot[color = 7blue1]
				table[x=t_tenpy, y=sz_chi_16, col sep=space]{figures/plots/TFI/global_quench/data/global_quench_g_critical_tenpy.txt};
				\addlegendentry{$\chi = 16$}
				%
				\addplot[color = 7blue2]
				table[x=t_tenpy, y=sz_chi_32, col sep=space]{figures/plots/TFI/global_quench/data/global_quench_g_critical_tenpy.txt};
				\addlegendentry{$\chi = 32$}
				%
				\addplot[color = 7blue3]
				table[x=t_tenpy, y=sz_chi_64, col sep=space]{figures/plots/TFI/global_quench/data/global_quench_g_critical_tenpy.txt};
				\addlegendentry{$\chi = 64$}
				%
				\addplot[color = 7blue4]
				table[x=t_tenpy, y=sz_chi_128, col sep=space]{figures/plots/TFI/global_quench/data/global_quench_g_critical_tenpy.txt};
				\addlegendentry{$\chi = 128$}
				%
				\addplot[color = 7blue5]
				table[x=t_tenpy, y=sz_chi_256, col sep=space]{figures/plots/TFI/global_quench/data/global_quench_g_critical_tenpy.txt};
				\addlegendentry{$\chi = 256$}
				%
				\addplot[color = 7blue6]
				table[x=t_tenpy, y=sz_chi_512, col sep=space]{figures/plots/TFI/global_quench/data/global_quench_g_critical_tenpy.txt};
				\addlegendentry{$\chi = 512$}
				%
				\addplot[color = 7blue7]
				table[x=t_tenpy, y=sz_chi_1024, col sep=space]{figures/plots/TFI/global_quench/data/global_quench_g_critical_tenpy.txt};
				\addlegendentry{$\chi = 1024$}
				%
			\end{axis}
			\begin{axis}[every axis plot/.append style={thick}, scale only axis, height=\singleFigureHeight, width=\singleFigureWidth, legend style={at={(1.015,0.9)}, anchor=north west, font=\small, nodes={scale=\legendscale, transform shape}, label={[font=\small]above:{disoTPS}}}, legend columns=1, clip mode=individual, legend cell align={left}, yticklabels=\empty, xmin=-0.05, xmax=1.05, ymin=0.78, ymax=1.02]
				%
				\addplot[color = 5red1]
				table[x=t_disoTPS, y=sz_disoTPS_D_2, col sep=space]{figures/plots/TFI/global_quench/data/global_quench_g_critical_disoTPS.txt};
				\addlegendentry{$D = 2$}
				%
				\addplot[color = 5red2]
				table[x=t_disoTPS, y=sz_disoTPS_D_3, col sep=space]{figures/plots/TFI/global_quench/data/global_quench_g_critical_disoTPS.txt};
				\addlegendentry{$D = 3$}
				%
				\addplot[color = 5red3]
				table[x=t_disoTPS, y=sz_disoTPS_D_4, col sep=space]{figures/plots/TFI/global_quench/data/global_quench_g_critical_disoTPS.txt};
				\addlegendentry{$D = 4$}
				%
				\addplot[color = 5red4]
				table[x=t_disoTPS, y=sz_disoTPS_D_5, col sep=space]{figures/plots/TFI/global_quench/data/global_quench_g_critical_disoTPS.txt};
				\addlegendentry{$D = 5$}
				%
				\addplot[color = 5red5]
				table[x=t_disoTPS, y=sz_disoTPS_D_6, col sep=space]{figures/plots/TFI/global_quench/data/global_quench_g_critical_disoTPS.txt};
				\addlegendentry{$D = 6$}
				%
			\end{axis}%
		\end{tikzpicture}%
	\end{minipage}
	\caption{In this figure we show the time evolution of the $\langle\hat{\sigma}_z\rangle$ expectation value of a spin in the middle of the $8\times8$ diagonal square lattice, containing in total $N = 128$ spins. As a model we use the TFI model at critical field $g_\text{C}$. We compute the time evolution once with DMRG on a MPS and once with disoTPS with the parameters given in the text. We observe good agreement up to times of $t\approx 0.2$, when the two disoTPS results diverge from the DMRG reference simulation.}
	\label{fig:disoTPS_time_evolution_g_critical}
\end{figure}
As a second experiment we study the capabilities of disoTPS to perform real-time evolution. For this we perform a global quench by initializing the disoTPS to a product state, which we then evolve in time, measuring local expectation values. We start with an all-up-state $|\Psi\rangle = |\uparrow\rangle\otimes\cdots\otimes|\uparrow\rangle$ on the square lattice and evolve with the Hamiltonian of the TFI model at critical transverse field $g \approx 3.04438$. At the critical field the entanglement is expected to grow very quickly, making this a hard problem. For the disoTPS we choose bond dimensions of $D\in\left\{2, 3, 4, 5, 6\right\}$, $\chi = 6\cdot D$ and a step size of $\Delta t = 0.02$. We evolve up to time $t = 1.0$, requiring 50 full TEBD iterations. For the YB-move we used approximate TRM disentangling optimizing the Rényi $\alpha = 0.5$ entropy for a maximum number of $N_\text{iter}^\text{YB} = 100$ iterations. For the application of the TEBD bond operators we used a maximum of $N_\text{iters}^\text{bond-op} = 100$ iterations. For a comparison we used the time evolution algorithm from \cite{cite:time_evolving_a_mps_with_long_range_interactions} defined on MPS, which is able to perform time evolution in the presence of long range interactions and is implemented in tenpy \cite{cite:tenpy}. For this reference simulation we used bond dimensions ranging from $\chi = 16$ to $\chi= 1024$ and a time step of $\Delta t = 0.01$. We show the results in figure \figref{fig:disoTPS_time_evolution_g_critical}. We observe that disoTPS is in good agreement with the reference simulation up to a time of $t\approx0.2$, at which the expectation value diverges. This happens at earlier times for smaller bond dimensions $D$. We expect that this fast divergence is due to the accumulated error of the YB move. One could improve the method by applying a per-column variational optimization as done in isoTPS \cite{cite:isometric_tensor_network_states_in_two_dimensions, cite:efficient_simulation_of_dynamics_in_two_dimensional_quantum_spin_systems}, which was found to be essential for real-time evolution \cite{cite:efficient_simulation_of_dynamics_in_two_dimensional_quantum_spin_systems}, or by improving the YB move. \par
%\usetikzlibrary{backgrounds} % DEBUG
%background rectangle/.style={fill=olive!45}, show background rectangle
\begin{figure}
	\centering
	\begin{minipage}[t]{1.0\textwidth}
		\hspace{20pt}
		\begin{tikzpicture}[scale=1, trim axis left, trim axis right]
			\begin{axis}[xlabel=$t$, ylabel=$\langle\hat{\sigma}_z\rangle$, grid=both, grid style={gray!20}, every axis plot/.append style={very thick}, scale only axis, height=\globalQuenchLargeFieldFigureHeight, width=\globalQuenchLargeFieldFigureWidth, xmin=-0.05, xmax=1.05, ymin=-1.1, ymax=1.1, legend style={nodes={scale=\legendscale, transform shape, font=\small}}, legend pos=south east, legend cell align={left}]
				%	
				\addplot[color = 5blue1]
				table[x=t_tenpy, y=sz_chi_16, col sep=space]{figures/plots/TFI/global_quench/data/global_quench_g_6.0_tenpy_site_index_4_4_0.txt};
				\addlegendentry{$\chi= 16$}
				%	
				\addplot[color = 5blue2]
				table[x=t_tenpy, y=sz_chi_32, col sep=space]{figures/plots/TFI/global_quench/data/global_quench_g_6.0_tenpy_site_index_4_4_0.txt};
				\addlegendentry{$\chi= 32$}
				%	
				\addplot[color = 5blue3]
				table[x=t_tenpy, y=sz_chi_64, col sep=space]{figures/plots/TFI/global_quench/data/global_quench_g_6.0_tenpy_site_index_4_4_0.txt};
				\addlegendentry{$\chi= 64$}
				%	
				\addplot[color = 5blue4]
				table[x=t_tenpy, y=sz_chi_128, col sep=space]{figures/plots/TFI/global_quench/data/global_quench_g_6.0_tenpy_site_index_4_4_0.txt};
				\addlegendentry{$\chi= 128$}
				%	
				\addplot[color = 5blue5]
				table[x=t_tenpy, y=sz_chi_256, col sep=space]{figures/plots/TFI/global_quench/data/global_quench_g_6.0_tenpy_site_index_4_4_0.txt};
				\addlegendentry{$\chi= 256$}
				%
			\end{axis}%
			\begin{axis}[scale only axis, height=\globalQuenchLargeFieldFigureHeight, width=\globalQuenchLargeFieldFigureWidth, every axis plot/.append style={very thick}, xmin=-0.05, xmax=1.05, ymin=-1.1, ymax=1.1, legend style={nodes={scale=\legendscale, transform shape, font=\small}}, legend pos=north west, legend cell align={left}]
				%	
				\addplot[color = 3red1]
				table[x=t_YB_isoTPS, y=sz_YB_isoTPS_D_2, col sep=space]{figures/plots/TFI/global_quench/data/global_quench_g_6.0_YB_isoTPS_site_index_4_4_0.txt};
				\addlegendentry{$D = 2$}
				%	
				\addplot[color = 3red2]
				table[x=t_YB_isoTPS, y=sz_YB_isoTPS_D_4, col sep=space]{figures/plots/TFI/global_quench/data/global_quench_g_6.0_YB_isoTPS_site_index_4_4_0.txt};
				\addlegendentry{$D = 4$}
				%	
				\addplot[color = 3red3]
				table[x=t_YB_isoTPS, y=sz_YB_isoTPS_D_6, col sep=space]{figures/plots/TFI/global_quench/data/global_quench_g_6.0_YB_isoTPS_site_index_4_4_0.txt};
				\addlegendentry{$D = 6$}
				%
			\end{axis}%
		\end{tikzpicture}%
		\quad
		\raisebox{34.2pt}
		{%
			\includegraphics[scale=1.1]{figures/tikz/TFI/site_indices/site_index_a.pdf}
		}
	\end{minipage}	
	\par\medskip
	\begin{minipage}{1.0\textwidth}
		\hspace{20pt}
		\begin{tikzpicture}[scale=1, trim axis left, trim axis right]
			\begin{axis}[xlabel=$t$, ylabel=$\langle\hat{\sigma}_z\rangle$, grid=both, grid style={gray!20}, every axis plot/.append style={very thick}, scale only axis, height=\globalQuenchLargeFieldFigureHeight, width=\globalQuenchLargeFieldFigureWidth, xmin=-0.05, xmax=1.05, ymin=0.5, ymax=1.1]
				%	
				\addplot[color = 5blue1]
				table[x=t_tenpy, y=sz_chi_16, col sep=space]{figures/plots/TFI/global_quench/data/global_quench_g_6.0_tenpy_site_index_4_4_1.txt};
				%\addlegendentry{$\chi= 16$}
				%	
				\addplot[color = 5blue2]
				table[x=t_tenpy, y=sz_chi_32, col sep=space]{figures/plots/TFI/global_quench/data/global_quench_g_6.0_tenpy_site_index_4_4_1.txt};
				%\addlegendentry{$\chi= 32$}
				%	
				\addplot[color = 5blue3]
				table[x=t_tenpy, y=sz_chi_64, col sep=space]{figures/plots/TFI/global_quench/data/global_quench_g_6.0_tenpy_site_index_4_4_1.txt};
				%\addlegendentry{$\chi= 64$}
				%	
				\addplot[color = 5blue4]
				table[x=t_tenpy, y=sz_chi_128, col sep=space]{figures/plots/TFI/global_quench/data/global_quench_g_6.0_tenpy_site_index_4_4_1.txt};
				%\addlegendentry{$\chi= 128$}
				%	
				\addplot[color = 5blue5]
				table[x=t_tenpy, y=sz_chi_256, col sep=space]{figures/plots/TFI/global_quench/data/global_quench_g_6.0_tenpy_site_index_4_4_1.txt};
				%\addlegendentry{$\chi= 256$}
				%
			\end{axis}%
			\begin{axis}[scale only axis, height=\globalQuenchLargeFieldFigureHeight, width=\globalQuenchLargeFieldFigureWidth, every axis plot/.append style={very thick}, xmin=-0.05, xmax=1.05, ymin=0.5, ymax=1.1]
				%	
				\addplot[color = 3red1]
				table[x=t_YB_isoTPS, y=sz_YB_isoTPS_D_2, col sep=space]{figures/plots/TFI/global_quench/data/global_quench_g_6.0_YB_isoTPS_site_index_4_4_1.txt};
				%\addlegendentry{$D = 2$}
				%	
				\addplot[color = 3red2]
				table[x=t_YB_isoTPS, y=sz_YB_isoTPS_D_4, col sep=space]{figures/plots/TFI/global_quench/data/global_quench_g_6.0_YB_isoTPS_site_index_4_4_1.txt};
				%\addlegendentry{$D = 4$}
				%	
				\addplot[color = 3red3]
				table[x=t_YB_isoTPS, y=sz_YB_isoTPS_D_6, col sep=space]{figures/plots/TFI/global_quench/data/global_quench_g_6.0_YB_isoTPS_site_index_4_4_1.txt};
				%\addlegendentry{$D = 5$}
				%
			\end{axis}%
		\end{tikzpicture}%
		\quad
		\raisebox{34.2pt}
		{%
			\includegraphics[scale=1.1]{figures/tikz/TFI/site_indices/site_index_b.pdf}
		}
	\end{minipage}
	\par\medskip
	\begin{minipage}{1.0\textwidth}
		\hspace{20pt}
		\begin{tikzpicture}[scale=1, trim axis left, trim axis right]
			\begin{axis}[xlabel=$t$, ylabel=$\langle\hat{\sigma}_z\rangle$, grid=both, grid style={gray!20}, every axis plot/.append style={very thick}, scale only axis, height=\globalQuenchLargeFieldFigureHeight, width=\globalQuenchLargeFieldFigureWidth, xmin=-0.05, xmax=1.05, ymin=0.9, ymax=1.05]
				%	
				\addplot[color = 5blue1]
				table[x=t_tenpy, y=sz_chi_16, col sep=space]{figures/plots/TFI/global_quench/data/global_quench_g_6.0_tenpy_site_index_5_5_0.txt};
				%\addlegendentry{$\chi= 16$}
				%	
				\addplot[color = 5blue2]
				table[x=t_tenpy, y=sz_chi_32, col sep=space]{figures/plots/TFI/global_quench/data/global_quench_g_6.0_tenpy_site_index_5_5_0.txt};
				%\addlegendentry{$\chi= 32$}
				%	
				\addplot[color = 5blue3]
				table[x=t_tenpy, y=sz_chi_64, col sep=space]{figures/plots/TFI/global_quench/data/global_quench_g_6.0_tenpy_site_index_5_5_0.txt};
				%\addlegendentry{$\chi= 64$}
				%	
				\addplot[color = 5blue4]
				table[x=t_tenpy, y=sz_chi_128, col sep=space]{figures/plots/TFI/global_quench/data/global_quench_g_6.0_tenpy_site_index_5_5_0.txt};
				%\addlegendentry{$\chi= 128$}
				%	
				\addplot[color = 5blue5]
				table[x=t_tenpy, y=sz_chi_256, col sep=space]{figures/plots/TFI/global_quench/data/global_quench_g_6.0_tenpy_site_index_5_5_0.txt};
				%\addlegendentry{$\chi= 256$}
				%
			\end{axis}%
			\begin{axis}[scale only axis, height=\globalQuenchLargeFieldFigureHeight, width=\globalQuenchLargeFieldFigureWidth, every axis plot/.append style={very thick}, xmin=-0.05, xmax=1.05, ymin=0.9, ymax=1.05]
				%	
				\addplot[color = 3red1]
				table[x=t_YB_isoTPS, y=sz_YB_isoTPS_D_2, col sep=space]{figures/plots/TFI/global_quench/data/global_quench_g_6.0_YB_isoTPS_site_index_5_5_0.txt};
				%\addlegendentry{$D = 2$}
				%	
				\addplot[color = 3red2]
				table[x=t_YB_isoTPS, y=sz_YB_isoTPS_D_4, col sep=space]{figures/plots/TFI/global_quench/data/global_quench_g_6.0_YB_isoTPS_site_index_5_5_0.txt};
				%\addlegendentry{$D = 4$}
				%	
				\addplot[color = 3red3]
				table[x=t_YB_isoTPS, y=sz_YB_isoTPS_D_6, col sep=space]{figures/plots/TFI/global_quench/data/global_quench_g_6.0_YB_isoTPS_site_index_5_5_0.txt};
				%\addlegendentry{$D = 5$}
				%
			\end{axis}%
		\end{tikzpicture}%
		\quad
		\raisebox{34.2pt}
		{%
			\includegraphics[scale=1.1]{figures/tikz/TFI/site_indices/site_index_c.pdf}
		}
	\end{minipage}
	\caption{In this figure we show the time evolution of the $\langle\hat{\sigma}_z\rangle$ expectation value of a spin in the middle of the lattice and its neighboring spins. The position of the measured spins is visualized in orange in the lattice next to the plots. As an initial state we choose the all-up state, with a spin in the center of the lattice flipped to the down-state $\ket{\downarrow}$. We draw the flipped spin in red. As a model we use the TFI model in the paramagnetic phase with a transverse field of $g = 6$, put on an $8\times8$ diagonal square lattice containing in total $N = 128$ spins. We compute the time evolution once with DMRG on a MPS and once with YB-isoTPS with the parameters given in the text.}
	\label{fig:YB_isoTPS_time_evolution_g_6}
\end{figure}
Finally, we want to test the time evolution in a less challenging regime, for which we used the TFI model at a transverse field of $g = 6$, which is well in the paramagnetic phase. In this regime, entanglement is expected to build up much slower than when simulating at the critical field. We again initialize an all-up-state $|\Psi\rangle = |\uparrow\rangle\otimes\cdots\otimes|\uparrow\rangle$ on the $8\times8$ square lattice but additionally flip a spin in the center. We then compute the time evolution of the $\langle\hat{\sigma}_z\rangle$ expectation value of the flipped center spin and neighbouring spins. The results are shown in figure \figref{fig:disoTPS_time_evolution_g_6}. Note that the DMRG reference simulation converges for much lower bond dimensions $\chi$ compared to figure \figref{fig:disoTPS_time_evolution_g_critical}. We also observe much better agreement of disoTPS TEBD and MPS DMRG.
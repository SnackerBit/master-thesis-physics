As a second experiment we studied the capabilities of disoTPS to perform real-time evolution. For this we perform a global quench by initializing the disoTPS to a product state, which we then evolve in time, measuring local expectation values. We start with an all-up-state $|\Psi\rangle = |\uparrow\rangle\otimes\cdots\otimes|\uparrow\rangle$ on the square lattice and evolve with the Hamiltonian of the TFI model at critical transverse field $g \approx 3.04438$. At the critical field the entanglement is expected to grow very quickly, making this a hard problem. For the disoTPS we choose bond dimensions of $D\in\left\{2, 3, 4, 5, 6\right\}$, $\chi = 6\cdot D$ and a step size of $\Delta t = 0.02$. We evolve up to time $t = 1.0$, requiring 50 full TEBD iterations. For the YB-move we used approximate TRM disentangling optimizing the Rényi $\alpha = 0.5$ entropy for a maximum number of $N_\text{iters}^\text{YB} = 100$ iterations. For the application of the TEBD bond operators we used a maximum of $N_\text{iters}^\text{bond-op} = 100$ iterations. For a comparison we used the time evolution algorithm from \cite{cite:time_evolving_a_mps_with_long_range_interactions} defined on MPS, which is able to perform time evolution in the presence of long range interactions and implemented in tenpy \cite{cite:tenpy}. For this we used bond dimensions ranging from $\chi = 16$ to $\chi= 1024$ and a time step of $\Delta t = 0.01$. We show the results in figure \figref{}.
The Density Matrix Renormalization Group (DMRG) algorithm, which was subsequently understood as a variational method over the class of Matrix Product States (MPS), has developed to be the de-facto standard for the numerical simulation of one-dimensional quantum systems. The success of this method is due to the remarkable ability of MPS to capture the area-law entanglement characteristics of ground states of gapped Hamiltonians. Additionally, due to the elegant diagrammatic notation for tensor networks, new algorithms can be developed and discussed efficiently and intuitively. Applications of MPS include finding ground and thermal states, real and imaginary time evolution, and the computation of dynamical properties of lattice Hamiltonians. In the following we give a brief introduction to MPS, for a more in-depth discussion see \cite{cite:DMRG_in_the_age_of_MPS, cite:practical_introduction_MPS_and_PEPS, cite:tenpy}. \par
The state of a quantum many-body system can be written as
\begin{equation}
	\left|\Psi\right\rangle = \sum_{i_1=1}^{d_1} \sum_{i_2=1}^{d_2} \cdots \sum_{i_N=1}^{d_N} \Psi_{i_1i_2\dots i_N} \left|i_1\right\rangle \otimes \left|i_2\right\rangle \otimes \cdots \otimes \left|i_N\right\rangle.
\end{equation}
where N is the number of subsystems (e.g. lattice sites or particles), and $\left\{\left|i_1\right\rangle \otimes \left|i_2\right\rangle \otimes \dots \otimes \left|i_N\right\rangle\right\}$, $i_j = 0, \dots, d_j$ is a set of basis vectors of the full many-body Hilbert space
\begin{equation}
	\mathcal{H} = \bigotimes_{j=1}^{N} \mathcal{H}_j,
\end{equation}
with $\dim\left(\mathcal{H}_j\right) = d_j$ the dimension of the local Hilbert space of subsystem $j$. To simplify the notation, we will assume that the dimension of all local subsystems is the same, $d_j = d$. The $d^N$ complex numbers $\Psi_{i_1i_2\dots i_N}$ fully describe the quantum many-body state, and one can think of $\Psi\in\mathbb{C}^{d\times\cdots\times d}$ as a tensor of rank $N$. However, due to the size of the tensor scaling exponentially with system size, only very small system sizes are accessible computationally. One proceeds by writing $\Psi$ as a tensor network of smaller tensors. A \textit{Matrix Product State} (MPS) is constructed by introducing $N$ rank-3 tensors $A^{[n]}\in\mathbb{C}^{d\times \chi_{n-1}\times \chi_{n}}$ and contracting them in a chain as
\begin{equation}
	\label{eq:MPS_open_boundary_conditions_general_definition}
	\Psi_{i_1i_2\cdots i_N} \coloneqq \sum_{\alpha_1=1}^{\chi_1} \sum_{\alpha_2=1}^{\chi_2}\dots\sum_{\alpha_{N-1}=1}^{\chi_{N-1}}A^{[1],i_1}_{1,\alpha_1} A^{[1],i_2}_{\alpha_1,\alpha_2} \cdots A^{[N],i_N}_{\alpha_{N-1},1},
\end{equation}
where we have written the physical indices $i_n$ as superscripts, such that the sums are performed only over subscripts. Note that in this notation the bond dimensions at the two ends of the chain is $\chi_0 = \chi_{N} = 1$, and we can interpret the tensors $A^{[1]}$ and $A^{[N]}$ as tensors of rank-2. A tensor diagram of the MPS \eqref{eq:MPS_open_boundary_conditions_general_definition} is given in figure \figref{fig:mps_general}.\par
\begin{figure}
	\centering
	\includegraphics[width=\textwidth]{figures/Tensor_Networks/mps_mps.jpeg}
	\caption{Diagrammatic representation of the Matrix Product State \ref{eq:MPS_open_boundary_conditions_general_definition}.}
	\label{fig:mps_general}
\end{figure}
\begin{figure}
	\centering
	\includegraphics[width=\textwidth]{figures/Tensor_Networks/mps_mps_canonical.jpeg}
	\caption{(a) Diagrammatic representation of an MPS in canonical form. (b) The isometry condition can be used to simplify contractions.}
	\label{fig:mps_canonical}
\end{figure}
An important property of MPS is the existence of a \textit{canonical form} as an isometric tensor network, where a single tensor $A^{[n]}$ is selected as the orthogonality center. One can bring an arbitrary MPS into this canonical form through successive QR-decompositions or SVDs, starting at the outer ends of the chain and isometrizing one tensor at a time, until the orthogonality center is reached \cite{cite:DMRG_in_the_age_of_MPS}. In figure \figref{fig:mps_canonical}(a) an MPS in canonical with the orthogonality center at subsystem $n$ is visualized in diagrammatic notation. The canonical form greatly simplifies many operations on MPS and allows for the formulation of efficient algorithms, where many contractions reduce to identity due to the isometry condition \eqref{eq:isometry_condition_general}, see also figure \figref{fig:mps_canonical}(b). For example, the expectation value $\left\langle\Psi\right|\hat{O}\left|\Psi\right\rangle$ of a one-site operator $\hat{O} \in \mathbb{C}^{d\times d}$ acting on site $n$ can for a general MPS be computed as
\begin{equation}
\begin{split}
	\label{eq:computation_of_expectation_value_MPS}
	\left\langle\Psi\right|\hat{O}\left|\Psi\right\rangle &=\sum_{i_1,\dots,i_N,j_n=1}^{d}\Psi_{i_1,i_2,\dots,i_N} \Psi_{i_1,\dots,i_{n-1},j_n,i_{n+1},\dots,i_N}^* \left\langle j_n\right|\hat{O} \left|i_n\right\rangle \\
	&= \sum_{i_1,\dots,i_N,j_n=1}^{d} \left(A^{[1],i_1}\cdots A^{[N],i_N}\right) \\
	&\quad\quad\quad\quad\quad\,\,\cdot\left(A^{[1],i_1*}\cdots A^{[N],j_n*} \cdots A^{[N],i_N*}\right)\cdot \hat{O}_{i_n,j_n},
\end{split}
\end{equation}
where the $A^{[n],i_n}$ are interpreted as matrices for $1 < n < N$ and as row/column vectors for $n = 1, N$ such that the product
\begin{equation}
	\left(A^{[1],i_1}\cdots A^{[N],i_N}\right)
\end{equation}
gives a scalar. The contraction \eqref{eq:computation_of_expectation_value_MPS} is visualized as a tensor diagram in figure \figref{fig:mps_local_expectation_value}. Here, the advantage of the diagrammatic notation becomes appearant: It is much easier to understand how tensors are contracted when expressing the contraction in terms of tensor network diagrams. The computational cost of computing the expectation value like this scales linear with the system size $\mathcal{O}\left(N\chi^3d\right)$, where $\chi$ is the maximum virtual bond dimension $\chi = \max\left\{\chi_1,\dots,\chi_N\right\}$. If the MPS is however given in canonical form with the orthogonality center at site $n$, the computation reduces to a contraction of only three tensors as can be seen in figure \figref{fig:mps_local_expectation_value_canonical}, and the computational cost $\mathcal{O}\left(\chi^3d\right)$ becomes independent of system size. \par
\begin{figure}
	\centering
	\includegraphics[width=\textwidth]{figures/Tensor_Networks/mps_local_expectation_value.jpeg}
	\caption{The computation of the expectation value of a local operator can be computed by contracting the MPS with its conjugate transpose, with the operator "sandwiched" between.}
	\label{fig:mps_local_expectation_value}
\end{figure}
\begin{figure}
	\centering
	\includegraphics[width=\textwidth]{figures/Tensor_Networks/mps_local_expectation_value_canonical.jpeg}
	\caption{If the MPS is in canonical form, the computation of the expectation value of a local operator can be simplified to a contraction of three tensors using the isometry condition.}
	\label{fig:mps_local_expectation_value_canonical}
\end{figure}
Until now, the MPS representation of $\left|\Psi\right\rangle$ is still exact. One can approximate a MPS by restricting the virtual bond dimension to a maximal bond dimension $\chi_n < \chi_\text{max}$. In this case, the number of parameters that need to be stored to describe the state is reduced from $\mathcal{O}\left(d^N\right)$ to $\mathcal{O}\left(N\chi_\text{max}^2 d\right)$. To arrive at this approximation, two neighbouring tensors can be contracted and split via a truncated SVD, keeping only the $\chi_\text{max}$ largest singular values. If the orthogonality center of the MPS is at one of the two tensors, this approximation is globally optimal as explained in section \ref{sec:tensors_and_tensor_networks_isometric_tensor_networks}. Additionally, this SVD at the orthogonality center is related to the Schimdt decomposition of a bipartite system
\begin{equation}
	\left|\Psi\right\rangle = \sum_{\alpha=1}^{\chi_n} \lambda_\alpha \left| \Psi^{[L]}_\alpha\right\rangle \otimes \left|\Psi^{[R]}_\alpha\right\rangle,
\end{equation}
where the chain is split into a left and right subsystem with orthogonal basis vectors $\left|\Psi^{[L]}_\alpha\right\rangle$ and $\left|\Psi^{[R]}_\alpha\right\rangle$ as visualized in figure \ref{fig}. In this case, the Schmidt values $\lambda_\alpha >= 0$ coincide with the singular values! One can further use this to compute the Von-Neumann entanglement entropy
\begin{equation}
	S = -\sum_{\alpha=1}^{\chi_n} \lambda_\alpha^2 \log\left(\lambda_\alpha^2\right),
\end{equation}
quantifying the amount of entanglement between the left and right subsystems. If the state is normalized, it additionally holds
\begin{equation}
	\sum_{\alpha=1}^{\chi_n} \lambda_\alpha^2 = 1.
\end{equation}
Thus, how well an MPS of a given bond dimension $\chi_\text{max}$ is able to represent a given quantum state is highly dependent on the Schmidt spectrum $\left\{\lambda_\alpha\right\}$ at the different bipartitions of the chain. If the Schmidt values decrease exponentially, only an exponentially small part of the entanglement structure is truncated and the truncated MPS is a good approximation for the original state. It can be shown \cite{cite:area_law_1D_proof, cite:area_laws_review} that for ground states of local, gapped, one dimensional Hamiltonians there holds an \textit{area law}: The entanglement entropy at arbitrary bipartitions of the chain is bounded by a constant
\begin{equation}
	S \le S_\text{max}, 
\end{equation}
where $S_\text{max}$ is independent of the system size. This is in contrast to the fact that the entanglement of states drawn randomly from the many-body Hilbert space on average exhibits \textit{volume law} scaling
\begin{equation}
	\mathbb{E}\left[S\right] > \min\left(N_L, N_R\right)\log(d),
\end{equation}
where $N_L$ and $N_R$ are the number of subsystems in the left and right bipartition. Hence, ground states of gapped Hamiltonians are very nongeneric. Note that the constant $S_\text{max}$ scales with the correlation length of the system, which diverges when approaching critical points. \par
It is immediately clear that truncated MPS by construction exhibit area law entanglement scaling, if the local subsystems that are represented by each tensor correspond to physical systems on a 1D chain. The maximal entanglement entropy for a bipartition can be reached when all Schmidt values are equal, $\lambda_\alpha = 1/\sqrt{\chi_n}$, and thus
\begin{equation}
	S \le \log\left(\chi_\text{max}\right)
\end{equation}
for arbitrary bipartitions of the chain. One can conclude that MPS are good approximations for ground states of gapped 1D Hamiltonians away from criticality. \par 
For completeness we note that the truncation of all bonds of an MPS is a highly non-linear optimization problem and the naive algorithm of truncating each bond with an SVD does in general not lead to a minimal error. A variational compression procedure can often be used to obtain a lower error at the same maximum bond dimension $\chi_\text{max}$ \cite{cite:DMRG_in_the_age_of_MPS}.
\par
\todo{algorithms}
In this thesis a new ansatz for the simulation of quantum lattice models in two dimensions was implemented, namely the diagonal isometric Tensor Product States (isoTPS). The ansatz is a variation of the isometric tensor product states (isoTPS) that generalize the canonical form of MPS to higher dimensions. In Chapter \ref{chap:tensors_and_tensor_networks} an introduction to tensor networks, isometric tensor networks, MPS and isoTPS was given. The new disoTPS were introduced in Chapter \ref{chap:disoTPS}. Several algorithms were discussed for moving the orthogonality hyper	surface through the disoTPS. It was found that a Riemannian disentangling procedure that minimizes the truncation error yields the lowest error for a single YB move. This was also found in previous works on isoTPS \cite{cite:isometric_tensor_network_states_in_two_dimensions, cite:efficient_simulation_of_dynamics_in_two_dimensional_quantum_spin_systems}. Further, an approximate algorithm was introduced for computing the required gradients and hessian vector products of the cost functions, decreasing the computational scaling from $\mathcal{O}(D^9)$ to $\mathcal{O}(D^8)$. A qualitative comparison of the different methods was performed, showing that the approximate versions of the algorithms perform almost as good as the exact versions, while being much faster. The introduced algorithms were then tested in chapters \ref{chap:toric_code} and \ref{chap:TFI}. In Chapter \ref{chap:toric_code} it was shown that disoTPS are able to represent the ground state of the Toric Code model, with the YB move being able to shift the orthogonality hypersurface without error up to numerical accuracy. In Chapter \ref{chap:TFI}, imaginary and real time evolution with the TEBD algorithm was used to find ground states and perform global quenches of the Transverse Field Ising model. It was found that disoTPS are able to produce similar results to isoTPS. The algorithms performing best in the ground state search were found to be the Riemannian optimization algorithms minimizing the Rényi entropy with $\alpha = 1/2$. In a large scale simulation of up to $N = 800$ spins, numerical evidence was found that disoTPS are able to correctly capture the area-law entanglement structure of the ground state. While at this regime a reference simulation of DMRG on MPS with a large bond dimension of $\chi = 1024$ was still able to produce lower ground state energies, it is expected that this will change when moving to larger system sizes or more challenging models. Additionally, the ansatz was used to compute the ground state energy of the TFI model on a honeycomb lattice, showcasing that disoTPS can be easily generalized to other lattice types. Computing a real time evolution after a global quench was found to be a very challenging problem for disoTPS. Because of the accumulation of YB errors, the time evolution diverges from the reference simulation after short times. This problem is less pronounced when moving away from the critical point, which corresponds to an easier problem because of the slower entanglement growth with time. Nevertheless, disoTPS was found to be able to correctly capture short-time behaviour of two-dimensional lattice systems of $N = 128$ spins even at criticality. \par
To summarize, the disoTPS ansatz is able to produce comparable results to isoTPS. While the computational complexity of shifting the orthogonality hypersurface of disoTPS is larger ($\mathcal{O}(D^8)$ compared to $\mathcal{O}(D^7)$) it is still an interesting alternative way of defining the isometry condition. One advantage of disoTPS is the ease with which the ansatz can be generalized to different lattices. For example, a generalization to the Kagome lattice should be possible without an increase in the cost of the YB move. Moreover, more involved algorithms like DMRG$^2$ require the contraction of a boundary MPS, which has a computational complexity scaling as $\mathcal{O}(D^{12})$. For such algorithms, we expect the increased cost for the YB move compared to the MM to not make a large difference. \par
It would be interesting to implement a DMRG$^2$ algorithm on disoTPS and compare it to isoTPS. Further work could also try to find alternative algorithms for performing the YB move, which we find to be the main source of errors. Similar to isoTPS, one could also implement a variational optimization on the complete column when shifting the orthogonality surface, which was found to be essential for good results when performing time evolution with isoTPS \cite{cite:efficient_simulation_of_dynamics_in_two_dimensional_quantum_spin_systems}. A further interesting research direction is the extension of the method to the thermodynamic limit. Lastly, the algorithm could benefit from an implementation on Graphics Processing Units (GPUs), which can speed up tensor contractions and decompositions drastically while also increasing the power efficiency.
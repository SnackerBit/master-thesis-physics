When discussing algorithms on isometric tensor networks, one often needs to find optimal tensors extremizing a given cost function $f$. In the most general case, $f$ is a function
\begin{equation}
	\label{eq:general_optimization_problem_of_isometric_tensor_networks_cost_function_multiple_input_tensors}
	f:\mathbb{C}^{m_1\times n_1}\times \dots \times \mathbb{C}^{m_K\times n_K} \to \mathbb{R},
\end{equation}
mapping $K$ tensors $T_1,\dots,T_K$ to a scalar cost value. Here, the tensors have already been reshaped into matrices, grouping together legs with incoming arrows and legs with outgoing arrows respectively. The tensors must satisfy certain constraints. If a tensor $T_i$ posesses both legs with incoming arrows and legs with outgoing arrows, it must satisfy the isometry constraint $T_i^\dagger T_i = \id$, where without loss of generalization we assumed $n_i \ge m_i$. If instead the tensor $T_j$ posesses only legs with incoming arrows (and thus is an orthogonality center), it is constrained to be normalized to one, $\lVert T_j\rVert_\text{F} = 1$. To summarize, we want to solve the optimization problem
\begin{equation}
	\label{eq:general_optimization_problem_of_isometric_tensor_networks_multiple_input_tensors}
	T_1^\text{opt}, \dots, T_K^\text{opt} = \underset{T_1,\dots T_K}{\text{argmax}}f\left(T_1, \dots T_K\right)
\end{equation}
under the constraints
\begin{equation}
	\label{eq:general_optimization_problem_of_isometric_tensor_networks_isometry_constraint}
	T_i^\dagger T_i = \id
\end{equation}
for isometries $T_i$ and
\begin{equation}
	\label{eq:general_optimization_problem_of_isometric_tensor_networks_ortho_center_constraint}
	\lVert T_j\rVert_\text{F} = 1
\end{equation}
for the orthogonality center $T_j$. \par
In the following, we will discuss several approaches for solving optimization problem \eqref{eq:general_optimization_problem_of_isometric_tensor_networks_multiple_input_tensors}. We will first assume that the input of the cost function is a single tensor $T$. If the cost function is linear, the problem is known as the \textit{orthogonal Procrustes problem} and we discuss its closed form solution in section \ref{sec:orthogonal_procrustes_problem}. Non-linear cost functions can be optimized by using the Evenbly-Vidal algorithm, see section \ref{sec:evenbly_vidal_algorithm}. Finally, we will discuss how cost functions of multiple tensors can be optimized in section \ref{sec:cost_functions_of_multiple_tensors}. \par
A different, more involved approach to solving the optimization problem is given by Riemannian optimization, which we discuss in appendix \ref{app:riemannian_optimization_of_isometries}.

\section{The orthogonal Procrustes problem}
\label{sec:orthogonal_procrustes_problem}
If the cost function is linear, it can be written as
\begin{equation}
	f(T) = \sum_{i=1}^{m}\sum_{j=1}^{n}\left[\alpha_{i,j}\Re\left(T_{i,j}\right) + \beta_{i,j} \Im\left(T_{i,j}\right)\right]
\end{equation}
with parameters $\alpha_{i,j}, \beta_{i,j} \in \mathbb{R}$. Introducing the \textit{environment tensor} $E\in\mathbb{C}^{m\times n}$ as $E_{i,j} = \alpha_i + \i \beta_j$ we can write the cost function as
\begin{equation}
	f(T) = \sum_{i=1}^{m}\sum_{j=1}^{n} \Re\left(E_{i,j}^*T_{i,j}\right) = \Re\Tr\left(E^\dagger T\right) = \Re\Tr\left(T^\dagger E\right).
\end{equation}
Maximizing $f(T)$ under the isometry constraint $T^\dagger T = \id$ is known as the orthogonal Procrustes problem and permits the closed form solution
\begin{equation}
	\label{eq:orthogonal_procrustes_problem_closed_form_solution}
	T^\text{opt} = \underset{T^\dagger T = \id}{\argmax} \Re\Tr\left(T^\dagger E\right) = UV^\dagger,
\end{equation}
where the matrices $U$ and $V$ are computed using an SVD $E = USV^\dagger$. To prove this result we insert the SVD into the cost function as
\begin{equation}
	\begin{split}
	f(T) &= \Re\Tr\left(ET^\dagger\right) = \Re\Tr\left(USV^\dagger T^\dagger\right) = \Re\Tr\left[\left(U\sqrt{S}\right)\left(\sqrt{S}V^\dagger T^\dagger\right)\right] \\
	&= \Re\left\langle\sqrt{S}U^\dagger,\sqrt{S}V^\dagger T^\dagger\right\rangle_\text{F}.
	\end{split}
\end{equation}
We next use the fact that the Frobenius inner product satisfies the Cauchy-Schwarz inequality to obtain the upper bound
\begin{equation}
	\begin{split}
		f(T) &= \Re\left\langle\sqrt{S}U^\dagger,\sqrt{S}V^\dagger T^\dagger\right\rangle_\text{F} \le \left\lVert\sqrt{S}U^\dagger\right\rVert_\text{F}\left\lVert\sqrt{S}V^\dagger T^\dagger\right\rVert_\text{F} \\
		&= \sqrt{\Tr\left(USU^\dagger\right)\Tr\left(TVSV^\dagger T^\dagger\right)} = \Tr\left(S\right),
	\end{split}
\end{equation}
where in the last step we used $U^\dagger U = \id$, $V^\dagger V = \id$, $T^\dagger T = \id$ and the cyclic property of the trace. This upper bound is reached by the solution
\begin{equation}
	F\left(T^\text{opt}\right) = \Re\Tr\left(USV^\dagger VU^\dagger\right) = \Tr\left(S\right),
\end{equation}
proving \eqref{eq:orthogonal_procrustes_problem_closed_form_solution}. \par
If the tensor $T$ must not satisfy the isometry condition but must be normalized to one, the closed form solution can be found as
\begin{equation}
	\label{eq:orthogonal_procrustes_problem_simple_case_closed_form_solution}
	T^\text{opt} = \underset{\left\lVert T\right\rVert = 1}{\argmax} \Re\Tr\left(T^\dagger E\right) = E/\left\lVert E\right\rVert.
\end{equation}
We arrive at this solution through a aimilar argument as before. First, we obtain an upper bound
\begin{equation}
	f(T) = \Re\Tr\left(T^\dagger E\right) = \Re\left\langle T, E\right\rangle_\text{F} \le \left\lVert T\right\rVert\left\lVert E\right\rVert = \left\lVert E\right\rVert
\end{equation}
using the Cauchy-Schwarz inequality and the normalization constraint $\left\lVert T\right\rVert = 1$. We proceed by showing that the upper bound is reached by $T^\text{opt}$,
\begin{equation}
	f(T^\text{opt}) = \Re\Tr\left(EE^\dagger/\left\lVert E\right\rVert\right) = \left\lVert E \right\rVert,
\end{equation}
proving \eqref{eq:orthogonal_procrustes_problem_simple_case_closed_form_solution}.

\section{The Evenbly-Vidal algorithm}
\label{sec:evenbly_vidal_algorithm}
In general the cost function $f(T)$ is not linear. For example, a non-linear cost function is encountered in the disentangling procedure when optimizing a MERA wave function \cite{}. It was proposed by Evenbly and Vidal \cite{} to linearize the cost function and to update the tensor $T$ iteratively using the closed form solutions from section \ref{sec:orthogonal_procrustes_problem}. Let us assume that the cost function $f(T)$ can be written as the contraction of a tensor network, where in general the tensor $T$ may appear multiple times. We contract all tensors except one of the tensors $T$ into an environment tensor $E_T \in\mathbb{C}^{n\times n}$ and the cost function becomes
\begin{equation}
	f(T)=\Re\Tr\left(E_TT\right).
\end{equation}
We now keep the environment $E_T$ fixed, trating it as if it were independant of $T$, and updating $T$ with the closed form solutions \eqref{eq:orthogonal_procrustes_problem_closed_form_solution} or \eqref{eq:orthogonal_procrustes_problem_simple_case_closed_form_solution}. This is repeated until $T$ is converged. If and how fast $T$ converges depends on the details of the cost fucntion, but connvergence cannot be guaranteed for arbitrary cost functions. This procedure is discussed in more detail and for general cost functions in appendix \ref{app:}\par
One can also use Riemannian optimization for the optimization of general non-linear cost functions of isometries. This method is more powerful but also more involved and is discussed in appendix \ref{app:riemannian_optimization_of_isometries}.

\section{Cost functions of multiple tensors}
\label{sec:cost_functions_of_multiple_tensors}
cost functions of multiple tensors $T_1, \dots T_K$ can be optimized iteratively via an algorithm similar to the Evenbly-Vidal algorithm. The idea is to optimize one tensor atr a time, keeping all other tensors fixed. To optimize the tensor $T_i$, we again contract all other tensors into an environment tensor $E$. If the environment tensor is independent of $T_i$ (i.e. if the cost function is linear in $T_i$), we can update the tensor with the closed form solutions of section \ref{sec:orthogonal_procrustes_problem}. Such an update is locally optimal in the sense that it maximizes $f(T_1, \dots, T_K)$ for fixed tensors $T_j$, $j\neq i$. If the environment tensor is dependant of $T_i$, we need to use the Evenbly-Vidal algorithm (see section \ref{sec:evenbly_vidal_algorithm}) or Riemannian optimization (see appendix \ref{app:riemannian_optimization_of_isometries}). If the cost function is linear in all tensors $T_1, \dots, T_K$ and bounded $f(T_1, \dots, T_K) \le c\in\mathbb{R}$, this algorithm is guaranteed to converge, since each local update is optimal and thus the cost function can never decrease. \par
An alternative approach for optimizing a cost function of multiple tensors is given by Riemannian optimization over product manifolds \cite{cite:riemannian_optimization_isometric_tensor_networks}, which we discuss briefly in appendix \ref{} \todo{Where do we discuss this?}
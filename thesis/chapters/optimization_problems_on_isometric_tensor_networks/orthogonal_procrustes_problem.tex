If the cost function is linear, it can be written as
\begin{equation}
	f(T) = \sum_{i=1}^{m}\sum_{j=1}^{n}\left[\alpha_{i,j}\Re\left(T_{i,j}\right) + \beta_{i,j} \Im\left(T_{i,j}\right)\right]
\end{equation}
with parameters $\alpha_{i,j}, \beta_{i,j} \in \mathbb{R}$. Introducing the \textit{environment tensor} $E\in\mathbb{C}^{m\times n}$ as $E_{i,j} = \alpha_i + \i \beta_j$ we can write the cost function as
\begin{equation}
	f(T) = \sum_{i=1}^{m}\sum_{j=1}^{n} \Re\left(E_{i,j}^*T_{i,j}\right) = \Re\Tr\left(E^\dagger T\right) = \Re\Tr\left(T^\dagger E\right).
\end{equation}
Maximizing $f(T)$ under the isometry constraint $T^\dagger T = \id$ is known as the orthogonal Procrustes problem and permits the closed form solution
\begin{equation}
	\label{eq:orthogonal_procrustes_problem_closed_form_solution}
	T^\text{opt} = \underset{T^\dagger T = \id}{\argmax} \Re\Tr\left(T^\dagger E\right) = UV^\dagger,
\end{equation}
where the matrices $U$ and $V$ are computed using an SVD $E = USV^\dagger$. To prove this result we insert the SVD into the cost function as
\begin{equation}
	\begin{split}
	f(T) &= \Re\Tr\left(ET^\dagger\right) = \Re\Tr\left(USV^\dagger T^\dagger\right) = \Re\Tr\left[\left(U\sqrt{S}\right)\left(\sqrt{S}V^\dagger T^\dagger\right)\right] \\
	&= \Re\left\langle\sqrt{S}U^\dagger,\sqrt{S}V^\dagger T^\dagger\right\rangle_\text{F}.
	\end{split}
\end{equation}
We next use the fact that the Frobenius inner product satisfies the Cauchy-Schwarz inequality to obtain the upper bound
\begin{equation}
	\begin{split}
		f(T) &= \Re\left\langle\sqrt{S}U^\dagger,\sqrt{S}V^\dagger T^\dagger\right\rangle_\text{F} \le \left\lVert\sqrt{S}U^\dagger\right\rVert_\text{F}\left\lVert\sqrt{S}V^\dagger T^\dagger\right\rVert_\text{F} \\
		&= \sqrt{\Tr\left(USU^\dagger\right)\Tr\left(TVSV^\dagger T^\dagger\right)} = \Tr\left(S\right),
	\end{split}
\end{equation}
where in the last step we used $U^\dagger U = \id$, $V^\dagger V = \id$, $T^\dagger T = \id$ and the cyclic property of the trace. This upper bound is reached by the solution
\begin{equation}
	F\left(T^\text{opt}\right) = \Re\Tr\left(USV^\dagger VU^\dagger\right) = \Tr\left(S\right),
\end{equation}
proving \eqref{eq:orthogonal_procrustes_problem_closed_form_solution}. \par
If the tensor $T$ must not satisfy the isometry condition but must be normalized to one, the closed form solution can be found as
\begin{equation}
	\label{eq:orthogonal_procrustes_problem_simple_case_closed_form_solution}
	T^\text{opt} = \underset{\left\lVert T\right\rVert = 1}{\argmax} \Re\Tr\left(T^\dagger E\right) = E/\left\lVert E\right\rVert.
\end{equation}
We arrive at this solution through a aimilar argument as before. First, we obtain an upper bound
\begin{equation}
	f(T) = \Re\Tr\left(T^\dagger E\right) = \Re\left\langle T, E\right\rangle_\text{F} \le \left\lVert T\right\rVert\left\lVert E\right\rVert = \left\lVert E\right\rVert
\end{equation}
using the Cauchy-Schwarz inequality and the normalization constraint $\left\lVert T\right\rVert = 1$. We proceed by showing that the upper bound is reached by $T^\text{opt}$,
\begin{equation}
	f(T^\text{opt}) = \Re\Tr\left(EE^\dagger/\left\lVert E\right\rVert\right) = \left\lVert E \right\rVert,
\end{equation}
proving \eqref{eq:orthogonal_procrustes_problem_simple_case_closed_form_solution}.
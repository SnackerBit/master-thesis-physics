\begin{figure}
	\centering
	\begin{tikzpicture}
		\centering
		\begin{axis}[xmin=0.5, xmax=12.5, xtick={1,2,3,4,5,6,7,8,9,10,11,12}, xticklabels={SVD, SVD + init, EV trunc, EV Rényi-2, CG trunc, approx CG trunc, TRM trunc, approx TRM trunc, CG Rényi-0.5, approx CG Rényi-0.5, TRM Rényi-0.5, approx TRM Rényi-0.5}, x tick label style={rotate=45, anchor=north east, inner sep=0mm}, ylabel={walltime [s]}, height=6cm, width=12cm, ymode=log, log origin=infty, ymajorgrids, yminorgrids]
			\addplot[ybar,bar width=0.6cm,fill=5blue4,draw=black,thick] plot coordinates{
				(1,0.9063506126403809)
				(2,0.7803809642791748)
				(3,5.042580604553223)
				(4,1.8616626262664795)
				(5,62.73412322998047)
				(7,224.94403100013733)
				(9,85.70946168899536)
				(11,451.5372350215912)};
			\addplot[ybar,bar width=0.6cm,fill=5blue3,draw=black, thick] plot coordinates{
				(6,22.153066873550415)
				(8,54.939692735672)
				(10,29.873801231384277)
				(12,73.2335991859436)};
		\end{axis}
	\end{tikzpicture}
	\caption{In this figure we compare the duration that the different algorithms spend on the YB move during a single TEBD time step (updating the complete disoTPS by an imaginary time $\Delta \tau$). We compute this walltime by running TEBD for $N_\text{TEBD} = 10$ steps, measuring the time spent performing YB moves, and computing the average walltime per TEBD step. We compare all methods benchmarked in figure \protect\figref{fig:tfi_gs_energy_vs_dtau_different_methods}. We plot the exact and approximate versions of thje Riemannian optimization algorithms next to each other and denote the approximate versions with a lighter color.}
	\label{fig:ground_state_search_barplots}
\end{figure}
\begin{figure}
	\centering
	\begin{minipage}{1.0\textwidth}
		\centering
		\begin{tikzpicture}[scale=1, trim axis left, trim axis right]
			\begin{axis}[xlabel=$\Delta\tau$, ylabel={$\Delta E / E_\text{exact}$}, grid=both, grid style={gray!20}, every axis plot/.append style={very thick}, scale only axis, height=\gsEnergyVsDtauFigureHeight, width=\gsEnergyVsDtauFigureWidth, xmode=log, ymode=log, ymin=1e-6, ymax=1e-1, title={$D_\text{horizontal} = D$}]
				%	
				\addplot[color = 3blue1, mark=*]
				table[x=dtau, y=delta_E_D_max_2, col sep=space]{figures/plots/TFI/gs_energy_vs_dtau/data/gs_energy_vs_dtau_honeycomb_svd_disent_renyi_0.5_approx_trm_D_horizontal_equal_D.txt};
				%\addlegendentry{$D = 2$}
				%	
				\addplot[color = 3blue2, mark=*]
				table[x=dtau, y=delta_E_D_max_4, col sep=space]{figures/plots/TFI/gs_energy_vs_dtau/data/gs_energy_vs_dtau_honeycomb_svd_disent_renyi_0.5_approx_trm_D_horizontal_equal_D.txt};
				%\addlegendentry{$D = 4$}
				%	
				\addplot[color = 3blue3, mark=*]
				table[x=dtau, y=delta_E_D_max_6, col sep=space]{figures/plots/TFI/gs_energy_vs_dtau/data/gs_energy_vs_dtau_honeycomb_svd_disent_renyi_0.5_approx_trm_D_horizontal_equal_D.txt};
				%\addlegendentry{$D = 6$}
			\end{axis}%
		\end{tikzpicture}%
		\,\,
		\begin{tikzpicture}[scale=1, trim axis left, trim axis right]
			\begin{axis}[xlabel=$\Delta\tau$, grid=both, grid style={gray!20}, every axis plot/.append style={very thick}, scale only axis, height=\gsEnergyVsDtauFigureHeight, width=\gsEnergyVsDtauFigureWidth, xmode=log, ymode=log, ymin=1e-6, ymax=1e-1, yticklabels={}, title={$D_\text{horizontal} = D^2$}, legend style={nodes={scale=\legendscale, transform shape, font=\small}}]
				%	
				\addplot[color = 3blue1, mark=*]
				table[x=dtau, y=delta_E_D_max_2, col sep=space]{figures/plots/TFI/gs_energy_vs_dtau/data/gs_energy_vs_dtau_honeycomb_svd_disent_renyi_0.5_approx_trm_D_horizontal_equal_D_squared.txt};
				\addlegendentry{$D = 2$}
				%	
				\addplot[color = 3blue2, mark=*]
				table[x=dtau, y=delta_E_D_max_4, col sep=space]{figures/plots/TFI/gs_energy_vs_dtau/data/gs_energy_vs_dtau_honeycomb_svd_disent_renyi_0.5_approx_trm_D_horizontal_equal_D_squared.txt};
				\addlegendentry{$D = 4$}
				%	
				\addplot[color = 3blue3, mark=*]
				table[x=dtau, y=delta_E_D_max_6, col sep=space]{figures/plots/TFI/gs_energy_vs_dtau/data/gs_energy_vs_dtau_honeycomb_svd_disent_renyi_0.5_approx_trm_D_horizontal_equal_D_squared.txt};
				\addlegendentry{$D = 6$}
			\end{axis}%
		\end{tikzpicture}%
		\quad\quad
		\raisebox{15pt}
		{%
			\includegraphics[scale=0.5]{figures/tikz/TFI/hexagonal_lattice/hexagonal_lattice_structure.pdf}
		}
	\end{minipage}
	\caption{In this figure we show imaginary TEBD results using YB-isoTPS on the honeycomb lattice. We used two different values for the horizontal bond dimension, $D_\text{horizontal} = D$ and $D_\text{horizontal} = D^2$. The bond dimension along the orthogonality hypersurface was chosen as $\chi = 6\cdot D$. For the YB move we used the approximate Rényi-$0.5$ disentangler with a maximum of $N_\text{iter} = 100$ iterations per YB move. The model is the TFI model on a $4\times 4$ honeycomb lattice at a transverse field of $g = 3.5$. On the right we show the YB-isoTPS structure of a $3\times 3$ honeycomb lattice as comparison.}
	\label{fig:tfi_gs_energy_vs_dtau_honeycomb}
\end{figure}
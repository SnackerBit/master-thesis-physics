\documentclass[crop, tikz]{standalone}
% This file is used to globally define colors, line widths, etc. for diagrams used in the thesis
\usepackage[intlimits]{amsmath}
\usepackage{amssymb}
\usepackage{tikz}
\usepackage{ifthen}
\usepackage{tikz-network}
\usepackage{pgfplots}
\pgfplotsset{compat=1.18}
\usepgfplotslibrary{colorbrewer}
\let\Re\relax
\DeclareMathOperator{\Re}{Re}
\DeclareMathOperator{\Tr}{Tr}

% Libraries
\usetikzlibrary{decorations.markings}
\usetikzlibrary{arrows}
\usetikzlibrary{positioning}
\usetikzlibrary{calc}

% Basic colors
\definecolor{borderColor}{rgb}{0.0, 0.0, 0.0}
\definecolor{physicalTensorColor}{rgb}{0.0, 0.0, 0.0}
\definecolor{auxillaryTensorColor}{rgb}{1.0, 0.0, 0.0}
\definecolor{orthogonalityCenterColor}{rgb}{1.0, 0.616, 0.0}
\definecolor{physicalLegColor}{rgb}{0.5, 0.5, 0.5}
\definecolor{auxillaryLegColor}{rgb}{1.0, 0.0, 0.0}
\definecolor{virtualLegColor}{rgb}{0.0, 0.0, 0.0}
\definecolor{generalTensorColor}{rgb}{0.757, 0.827, 0.996}
\definecolor{identityColor}{rgb}{1.0, 1.0, 1.0}
\definecolor{redMarkingColor}{rgb}{1.0, 0.0, 0.0}
\definecolor{greenMarkingColor}{rgb}{0.1, 0.8, 0.1}
\definecolor{operatorColor}{rgb}{0.5, 1.0, 0.5}
\definecolor{specialLegColor}{HTML}{5d3fd3}
\definecolor{toricCodeSpinColor}{rgb}{1.0, 0.0, 0.0}
\definecolor{toricCodeEdgeSpinColor}{rgb}{0.0, 0.0, 0.7}
\definecolor{toricCodeAOperatorColor}{rgb}{0.0, 0.8, 0.0}
\definecolor{toricCodeBOperatorColor}{rgb}{0.0, 0.0, 1.0}
\definecolor{toricCodeDisoTPSTensorColor}{HTML}{5d3fd3}
\def \backgroundopacity {0.5}

\definecolor{color1}{HTML}{3182bd} % Same as 5blue4
\definecolor{color2}{HTML}{fd8d3c} % Same as 5orange3
\definecolor{color3}{HTML}{31a354} % Same as 5green4
\definecolor{color4}{HTML}{a50f15} % Same as 5red5

% Basic sizes
\def \defaultTensorWidth {19pt}
\def \smallTensorWidth {13pt}
\def \tinyTensorWidth {4pt}
\def \defaultLineWidth {2}
\def \lineWidthThick {2.5}
\def \lineWidthThin {1.8}
\def \lineWidthTiny {1.2}
\def \lineWidthHuge {4}
\def \xOffsetUnitCell {180pt}
\def \yOffsetUnitCell {180pt}
\def \yOffsetPhysicalLeg {30pt}
\def \physicalLegLength {30pt}
\def \physicalLegLengthSmall {15pt}
\def \defaultArrowscale {0.7}
\def \arrowScaleSmall {0.6}
\def \defaultArrowXShift {7pt}
\def \defaultDoubleArrowXShift {5pt}
\def \defaultTextOffset {6pt}
\def \defaultTextOffsetLarge {9pt}
\def \defaultTextOffsetSmall {1pt}
\def \defaultDistanceSmall {30pt}
\def \defaultDistanceSmallDiagonal {32pt}
\def \defaultDistanceNormal {40pt}
\def \defaultDistanceLarge {62 pt}
\def \defaultDistanceEquations {40pt}
\def \identityLegDistance {10 pt}
\def \isoTPSPhysicalLegAngle {30}
\def \defaultLegSeperationSmall{4pt}
\def \defaultLegSeperation{6.5pt}
\def \defaultLegSeperationLarge{9pt}
\def \roundedCornerInsetNormal{10pt}
\def \roundedCornerInsetSmall{2pt}
\def \braceWidth{10pt}
\def \physicalLegLengthSmallMPS{1.3*\physicalLegLengthSmall}


\def\tensorDistanceMPS{1.2*\defaultDistanceSmall}
\def\braketDistanceMPS{1.2*\defaultDistanceNormal}
\def\tensorWidthMPS{\smallTensorWidth}
\def\openLegMultiplierMPS{0.9}

% Tensor styles
\tikzstyle{tensorPhysical} = [circle, minimum size=\defaultTensorWidth, line width=\defaultLineWidth, fill=physicalTensorColor, draw=borderColor, outer sep=0]
\tikzstyle{tensorAuxillary} = [circle, minimum size=\defaultTensorWidth, line width=\defaultLineWidth, fill=auxillaryTensorColor, draw=borderColor, outer sep=0]
\tikzstyle{tensorOrthoCenter} = [circle, minimum size=\defaultTensorWidth, line width=\defaultLineWidth, fill=orthogonalityCenterColor, draw=borderColor, outer sep=0]

\tikzstyle{tensorPhysicalSmall} = [circle, minimum size=\smallTensorWidth, line width=\defaultLineWidth, fill=physicalTensorColor, draw=borderColor, outer sep=0]
\tikzstyle{tensorAuxillarySmall} = [circle, minimum size=\smallTensorWidth, line width=\defaultLineWidth, fill=auxillaryTensorColor, draw=borderColor, outer sep=0]
\tikzstyle{tensorOrthoCenterSmall} = [circle, minimum size=\smallTensorWidth, line width=\defaultLineWidth, fill=orthogonalityCenterColor, draw=borderColor, outer sep=0]

\tikzstyle{tensorOperator} = [rectangle, line width=\defaultLineWidth, fill=operatorColor, draw=borderColor, rounded corners=\roundedCornerInsetNormal]
\tikzstyle{tensorOperatorSmall} = [rectangle, line width=\defaultLineWidth, fill=operatorColor, draw=borderColor, rounded corners=\roundedCornerInsetSmall]

\tikzstyle{toricCodeSpin} = [circle, minimum size=\tinyTensorWidth, line width=\defaultLineWidth, fill=toricCodeSpinColor, draw=toricCodeSpinColor, inner sep=0, outer sep=0]
\tikzstyle{toricCodeEdgeSpin} = [circle, minimum size=\tinyTensorWidth, line width=\defaultLineWidth, fill=toricCodeEdgeSpinColor, draw=toricCodeEdgeSpinColor, inner sep=0, outer sep=0]
\tikzstyle{toricCodePEPSCenterTensor} = [circle, minimum size=2*\tinyTensorWidth, line width=\defaultLineWidth, fill=physicalTensorColor, draw=physicalTensorColor, inner sep=0, outer sep=0]
\tikzstyle{toricCodeDisoTPSTensor} = [circle, minimum size=0.9*\defaultTensorWidth, line width=\defaultLineWidth, fill=toricCodeDisoTPSTensorColor, draw=physicalTensorColor, inner sep=0, outer sep=0]

\tikzstyle{tensorPhysicalMPS} = [circle, minimum size=\smallTensorWidth, line width=\defaultLineWidth, fill=generalTensorColor, draw=borderColor, outer sep=0]
\tikzstyle{tensorPhysicalMPSCanonical} = [circle, minimum size=\smallTensorWidth, line width=\defaultLineWidth, fill=physicalTensorColor, draw=borderColor, outer sep=0]
\tikzstyle{tensorPhysicalMPSCanonicalOrthoCenter} = [circle, minimum size=\smallTensorWidth, line width=\defaultLineWidth, fill=orthogonalityCenterColor, draw=borderColor, outer sep=0]

% Leg styles
\tikzstyle{auxillaryLeg} = [color=auxillaryLegColor, line width=\defaultLineWidth, decoration={markings, mark=at position 0.5 with {\pgftransformscale{\defaultArrowscale}\arrow[xshift=\defaultArrowXShift]{triangle 60}}}, postaction={decorate}]
\tikzstyle{physicalLeg} = [color=physicalLegColor, line width=\defaultLineWidth, decoration={markings, mark=at position 0.5 with {\pgftransformscale{\defaultArrowscale}\arrow[xshift=\defaultArrowXShift]{triangle 60}}}, postaction={decorate}]
\tikzstyle{virtualLeg} = [color=virtualLegColor, line width=\defaultLineWidth, decoration={markings, mark=at position 0.5 with {\pgftransformscale{\defaultArrowscale}\arrow[xshift=\defaultArrowXShift]{triangle 60}}}, postaction={decorate}]
\tikzstyle{specialLeg} = [color=specialLegColor, line width=\defaultLineWidth, decoration={markings, mark=at position 0.5 with {\pgftransformscale{\defaultArrowscale}\arrow[xshift=\defaultArrowXShift]{triangle 60}}}, postaction={decorate}]

\tikzstyle{virtualLegLateArrows} = [color=virtualLegColor, line width=\defaultLineWidth, decoration={markings, mark=at position 0.8 with {\pgftransformscale{\defaultArrowscale}\arrow[xshift=\defaultArrowXShift]{triangle 60}}}, postaction={decorate}]
\tikzstyle{virtualLegWithoutArrows} = [color=virtualLegColor, line width=\defaultLineWidth]
\tikzstyle{physicalLegWithoutArrows} = [color=physicalLegColor, line width=\defaultLineWidth]
\tikzstyle{specialLegWithoutArrows} = [color=specialLegColor, line width=\defaultLineWidth]
\tikzstyle{virtualLegDoubleArrow} = [color=virtualLegColor, line width=\defaultLineWidth, decoration={markings, mark=at position 0.5 with {\pgftransformscale{\defaultArrowscale}\arrow[xshift=\defaultDoubleArrowXShift]{Rays[round]}}}, postaction={decorate}]
\tikzstyle{auxillaryLegWithoutArrows} = [color=auxillaryLegColor, line width=\defaultLineWidth]

\tikzstyle{auxillaryLegSmall} = [color=auxillaryLegColor, line width=\lineWidthThin, decoration={markings, mark=at position 0.5 with {\pgftransformscale{\arrowScaleSmall}\arrow[xshift=\defaultArrowXShift]{triangle 60}}}, postaction={decorate}]
\tikzstyle{auxillaryLegSmallWithoutArrows} = [color=auxillaryLegColor, line width=\lineWidthThin]
\tikzstyle{auxillaryLegSmallDoubleArrows} = [color=auxillaryLegColor, line width=\lineWidthThin, decoration={markings, mark=at position 0.5 with {\pgftransformscale{\arrowScaleSmall}\arrow[xshift=\defaultDoubleArrowXShift]{Rays[round]}}}, postaction={decorate}]
\tikzstyle{physicalLegSmall} = [color=physicalLegColor, line width=\lineWidthThin, decoration={markings, mark=at position 0.5 with {\pgftransformscale{\arrowScaleSmall}\arrow[xshift=\defaultArrowXShift]{triangle 60}}}, postaction={decorate}]
\tikzstyle{virtualLegSmall} = [color=virtualLegColor, line width=\lineWidthThin, decoration={markings, mark=at position 0.5 with {\pgftransformscale{\arrowScaleSmall}\arrow[xshift=\defaultArrowXShift]{triangle 60}}}, postaction={decorate}]
\tikzstyle{virtualLegSmallWithoutArrows} = [color=virtualLegColor, line width=\lineWidthThin]
\tikzstyle{physicalLegSmallWithoutArrows} = [color=physicalLegColor, line width=\lineWidthThin]
\tikzstyle{virtualLegSmallDoubleArrows} = [color=virtualLegColor, line width=\lineWidthThin, decoration={markings, mark=at position 0.5 with {\pgftransformscale{\arrowScaleSmall}\arrow[xshift=\defaultDoubleArrowXShift]{Rays[round]}}}, postaction={decorate}]

% special symbols
\usepackage{amssymb}
\DeclareMathAlphabet{\mathbbb}{U}{bbold}{m}{n}
\newcommand{\id}{\mathbbb{1}}
\newcommand{\iu}{\mathrm{i}}% imaginary unit number i
\newcommand{\Stiefel}{\text{St}(n,p)}


% Layers
\pgfdeclarelayer{bg} % declare background layer
\pgfsetlayers{bg,main} % set order of layers

% Command for drawing convex hulls
\newcommand{\convexhull}[2]{
	[   
	create hullnodes/.code={
		\global\edef\namelist{#1}
		\foreach [count=\counter] \nodename in \namelist {
			\global\edef\numberofnodes{\counter}
			\node at (\nodename) [draw=none,name=hullnode\counter] {};
		}
		\node at (hullnode\numberofnodes) [name=hullnode0,draw=none] {};
		\pgfmathtruncatemacro\lastnumber{\numberofnodes+1}
		\node at (hullnode1) [name=hullnode\lastnumber,draw=none] {};
	},
	create hullnodes
	]
	($(hullnode1)!#2!-90:(hullnode0)$)
	\foreach [
	evaluate=\currentnode as \previousnode using \currentnode-1,
	evaluate=\currentnode as \nextnode using \currentnode+1
	] \currentnode in {1,...,\numberofnodes} {
		-- ($(hullnode\currentnode)!#2!-90:(hullnode\previousnode)$)
		let \p1 = ($(hullnode\currentnode)!#2!-90:(hullnode\previousnode) - (hullnode\currentnode)$),
		\n1 = {atan2(\y1,\x1)},
		\p2 = ($(hullnode\currentnode)!#2!90:(hullnode\nextnode) - (hullnode\currentnode)$),
		\n2 = {atan2(\y2,\x2)},
		\n{delta} = {-Mod(\n1-\n2,360)}
		in 
		{arc [start angle=\n1, delta angle=\n{delta}, radius=#2]}
	}
	-- cycle
}

% Command for drawing braces
%% https://tex.stackexchange.com/questions/55068/is-there-a-tikz-equivalent-to-the-pstricks-ncbar-command
\tikzset{
	ncbar angle/.initial=90,
	ncbar/.style={
		to path=(\tikztostart)
		-- ($(\tikztostart)!#1!\pgfkeysvalueof{/tikz/ncbar angle}:(\tikztotarget)$)
		-- ($(\tikztotarget)!($(\tikztostart)!#1!\pgfkeysvalueof{/tikz/ncbar angle}:(\tikztotarget)$)!\pgfkeysvalueof{/tikz/ncbar angle}:(\tikztostart)$)
		-- (\tikztotarget)
	},
	ncbar/.default=0.5cm,
}

\tikzset{square left brace/.style={ncbar=10pt}}
\tikzset{square right brace/.style={ncbar=-10pt}}

\tikzset{round left paren/.style={ncbar=10pt,out=120,in=-120}}
\tikzset{round right paren/.style={ncbar=10pt,out=60,in=-60}}

% PLOTS
\RequirePackage{pgfplots}
\pgfplotsset{compat=1.18}
\usepgfplotslibrary{colorbrewer}
\RequirePackage{tikz}
% A single subfigure
\def\singleFigureWidth{8 cm}
\def\singleFigureHeight{4.944 cm}
\def\insetFigureWidth{4 cm}
\def\insetFigureHeight{2.2 cm}
% Two subfigures stacked ontop of each other
\def\doubleVerticalFigureWidth{11 cm}
\def\doubleVerticalFigureHeight{5 cm}

% 3 subfigures on top of each other
\def\tripleVerticalFigureWidth{11 cm}
\def\tripleVerticalFigureHeight{4 cm}

% 4 subfigures aranged in a 2x2 pattern
\def\twoByTwoFigureWidth{6.75cm}
\def\twoByTwoFigureHeight{4.1717cm}

% 4 subfigures side by side on one page
\def\gsEnergyVsDtauFigureWidth{3.2cm}
\def\gsEnergyVsDtauFigureHeight{4cm}

% 3 lengthy subfigures below each other
\def\globalQuenchLargeFieldFigureWidth{10cm}
\def\globalQuenchLargeFieldFigureHeight{3cm}

\def\legendscale{1.0}

\def\doubleFigureWidth{6.125cm}
\def\doubleFigureHeight{4.635cm}
% PLOT Colors
\definecolor{7blue1}{HTML}{eff3ff}
\definecolor{7blue2}{HTML}{c6dbef}
\definecolor{7blue3}{HTML}{9ecae1}
\definecolor{7blue4}{HTML}{6baed6}
\definecolor{7blue5}{HTML}{4292c6}
\definecolor{7blue6}{HTML}{2171b5}
\definecolor{7blue7}{HTML}{084594}

\definecolor{5blue1}{HTML}{eff3ff}
\definecolor{5blue2}{HTML}{bdd7e7}
\definecolor{5blue3}{HTML}{6baed6}
\definecolor{5blue4}{HTML}{3182bd}
\definecolor{5blue5}{HTML}{08519c}

\definecolor{3blue1}{HTML}{6baed6}
\definecolor{3blue2}{HTML}{3182bd}
\definecolor{3blue3}{HTML}{08519c}

\definecolor{5orange1}{HTML}{feedde}
\definecolor{5orang2}{HTML}{fdbe85}
\definecolor{5orange3}{HTML}{fd8d3c}
\definecolor{5orange4}{HTML}{e6550d}
\definecolor{5orange5}{HTML}{a63603}

\definecolor{5red1}{HTML}{fee5d9}
\definecolor{5red2}{HTML}{fcae91}
\definecolor{5red3}{HTML}{fb6a4a}
\definecolor{5red4}{HTML}{de2d26}
\definecolor{5red5}{HTML}{a50f15}

\definecolor{3red1}{HTML}{fb6a4a}
\definecolor{3red2}{HTML}{de2d26}
\definecolor{3red3}{HTML}{a50f15}

\definecolor{4red1}{HTML}{fee5d9}
\definecolor{4red2}{HTML}{fcae91}
\definecolor{4red3}{HTML}{fb6a4a}
\definecolor{4red4}{HTML}{cb181d}

\definecolor{5green1}{HTML}{edf8e9}
\definecolor{5green2}{HTML}{bae4b3}
\definecolor{5green3}{HTML}{74c476}
\definecolor{5green4}{HTML}{31a354}
\definecolor{5green5}{HTML}{006d2c}

\definecolor{5gray1}{HTML}{f7f7f7}
\definecolor{5gray2}{HTML}{cccccc}
\definecolor{5gray3}{HTML}{969696}
\definecolor{5gray4}{HTML}{636363}
\definecolor{5gray5}{HTML}{252525}

\definecolor{5purple1}{HTML}{f2f0f7}
\definecolor{5purple2}{HTML}{cbc9e2}
\definecolor{5purple3}{HTML}{9e9ac8}
\definecolor{5purple4}{HTML}{756bb1}
\definecolor{5purple5}{HTML}{54278f}

\definecolor{singleBlue}{HTML}{3182bd} % Same as 5blue4
\definecolor{singleOrange}{HTML}{fd8d3c} % Same as 5orange3
\definecolor{singleGreen}{HTML}{31a354} % Same as 5green4
\definecolor{singleRed}{HTML}{a50f15} % Same as 5red5

% axis style, ticks, etc
\pgfplotsset{every axis/.append style={
		label style={font=\small},
		tick label style={font=\small}}}
\usepackage{tikz-network}

\begin{document}
	\def\openLegLength{21 pt}
	\def\tensorDistance{1.0*\defaultDistanceNormal}
	\def\ccdistance{70 pt}
	
	\def\posXNA{0}
	\def\posYNA{-\ccdistance/2}
	\def\posXNB{\tensorDistance*0.8}
	\def\posYNB{\tensorDistance*0.3-\ccdistance/2}
	\def\posXNC{\tensorDistance*0.5}
	\def\posYNC{-\tensorDistance*0.4-\ccdistance/2}
	\def\posXND{\tensorDistance*1.2}
	\def\posYND{-\tensorDistance*0.5-\ccdistance/2}
	\def\posXNE{\tensorDistance*1.8}
	\def\posYNE{-\tensorDistance*0.45-\ccdistance/2}
	\def\posXNF{\tensorDistance*1.5}
	\def\posYNF{\tensorDistance*0.4-\ccdistance/2}
	\def\posXNG{\tensorDistance*2.3}
	\def\posYNG{-\tensorDistance*0.1-\ccdistance/2}
	\def\posXNH{\tensorDistance*2.2}
	\def\posYNH{-\tensorDistance*0.8-\ccdistance/2}
	\def\posXNI{\tensorDistance*2.9}
	\def\posYNI{-\tensorDistance*0.2-\ccdistance/2}
	\def\posXNJ{\tensorDistance*3.1}
	\def\posYNJ{\tensorDistance*0.4-\ccdistance/2}
	
	\def\posXNpA{0}
	\def\posYNpA{\ccdistance/2}
	\def\posXNpB{\tensorDistance*0.8}
	\def\posYNpB{\tensorDistance*0.3+\ccdistance/2}
	\def\posXNpC{\tensorDistance*0.5}
	\def\posYNpC{-\tensorDistance*0.4+\ccdistance/2}
	\def\posXNpF{\tensorDistance*1.5}
	\def\posYNpF{\tensorDistance*0.4+\ccdistance/2}
	\def\posXNpG{\tensorDistance*2.3}
	\def\posYNpG{-\tensorDistance*0.1+\ccdistance/2}
	\def\posXNpH{\tensorDistance*2.2}
	\def\posYNpH{-\tensorDistance*0.8+\ccdistance/2}
	\def\posXNpI{\tensorDistance*2.9}
	\def\posYNpI{-\tensorDistance*0.2+\ccdistance/2}
	\def\posXNpJ{\tensorDistance*3.1}
	\def\posYNpJ{\tensorDistance*0.4+\ccdistance/2}
	\def\posXNpK{\tensorDistance*1.1}
	\def\posYNpK{-\tensorDistance*0.55+\ccdistance/2}
	\def\posXNpL{\tensorDistance*1.7}
	\def\posYNpL{-\tensorDistance*0.6+\ccdistance/2}
	\def\posXNpM{\tensorDistance*1.45}
	\def\posYNpM{-\tensorDistance*0.15+\ccdistance/2}
	
	\def\angleA{170}
	\def\angleC{-120}
	\def\angleF{35}
	\def\angleH{-45}
	\def\angleI{-10}
	\def\angleJ{20}
	
	\begin{tikzpicture}
		% Draw beginning of equation and first equal sign
		\node[] () at ({\posXNA+\openLegLength*cos(\angleA)-\defaultDistanceEquations/2+7pt}, 0) {$=$};
		\node[] () at ({\posXNA+\openLegLength*cos(\angleA)-\defaultDistanceEquations-5pt}, 0) {$\langle\mathcal{N},\mathcal{N}^\prime\rangle_\text{F}$};
		
		% Draw tensors of N
		\node[tensorPhysical, minimum size=\smallTensorWidth] (NA) at (\posXNA, \posYNA) {};
		\node[tensorPhysical, minimum size=\smallTensorWidth] (NB) at (\posXNB, \posYNB) {};
		\node[tensorPhysical, minimum size=\smallTensorWidth] (NC) at (\posXNC, \posYNC) {};
		\node[tensorOrthoCenter, minimum size=\smallTensorWidth] (ND) at (\posXND, \posYND) {};
		\node[tensorPhysical, minimum size=\smallTensorWidth, fill=redMarkingColor] (NE) at (\posXNE, \posYNE) {};
		\node[tensorPhysical, minimum size=\smallTensorWidth] (NF) at (\posXNF, \posYNF) {};
		\node[tensorPhysical, minimum size=\smallTensorWidth] (NG) at (\posXNG, \posYNG) {};
		\node[tensorPhysical, minimum size=\smallTensorWidth] (NH) at (\posXNH, \posYNH) {};
		\node[tensorPhysical, minimum size=\smallTensorWidth] (NI) at (\posXNI, \posYNI) {};
		\node[tensorPhysical, minimum size=\smallTensorWidth] (NJ) at (\posXNJ, \posYNJ) {};
		
		\begin{pgfonlayer}{bg}
			% Draw subregion M
			%\fill[fill=redMarkingColor, opacity=\backgroundopacity] \convexhull{D, E}{\smallTensorWidth*0.9};
			%\node[color=redMarkingColor] (M) at (\posXND-\tensorDistance*0.2, \posYND-\tensorDistance*0.4) {$\mathcal{M}$};
			
			% Draw connecting legs
			\draw[virtualLeg] (NA) -- (NB);
			\draw[virtualLeg] (NB) -- (NF);
			\draw[virtualLeg] (NC) -- (NB);
			\draw[virtualLeg] (NC) -- (ND);
			\draw[virtualLeg] (NE) -- (ND);
			\draw[virtualLeg] (NF) -- (NG);
			\draw[virtualLeg] (NF) -- (NE);
			\draw[virtualLeg] (NG) -- (NH);
			\draw[virtualLeg] (NH) -- (NE);
			\draw[virtualLeg] (NI) -- (NJ);
			\draw[virtualLeg] (NI) -- (NG);
			\draw[virtualLeg] (NJ) -- (NG);
			
			\draw[virtualLeg] ({\posXNA+\openLegLength*cos(\angleA)}, {\posYNA+\openLegLength*sin(\angleA)}) -- (NA);
			\draw[virtualLeg] ({\posXNC+\openLegLength*cos(\angleC)}, {\posYNC+\openLegLength*sin(\angleC)}) -- (NC);
			\draw[virtualLeg] ({\posXNF+\openLegLength*cos(\angleF)}, {\posYNF+\openLegLength*sin(\angleF)}) -- (NF);
			\draw[virtualLeg] ({\posXNH+\openLegLength*cos(\angleH)}, {\posYNH+\openLegLength*sin(\angleH)}) -- (NH);
			\draw[virtualLeg] ({\posXNI+\openLegLength*cos(\angleI)}, {\posYNI+\openLegLength*sin(\angleI)}) -- (NI);
			\draw[virtualLeg] ({\posXNJ+\openLegLength*cos(\angleJ)}, {\posYNJ+\openLegLength*sin(\angleJ)}) -- (NJ);
		\end{pgfonlayer}
		
		% Draw tensors of N'
		\node[tensorPhysical, minimum size=\smallTensorWidth] (NpA) at (\posXNpA, \posYNpA) {};
		\node[tensorPhysical, minimum size=\smallTensorWidth] (NpB) at (\posXNpB, \posYNpB) {};
		\node[tensorPhysical, minimum size=\smallTensorWidth] (NpC) at (\posXNpC, \posYNpC) {};
		\node[tensorPhysical, minimum size=\smallTensorWidth] (NpF) at (\posXNpF, \posYNpF) {};
		\node[tensorPhysical, minimum size=\smallTensorWidth] (NpG) at (\posXNpG, \posYNpG) {};
		\node[tensorPhysical, minimum size=\smallTensorWidth] (NpH) at (\posXNpH, \posYNpH) {};
		\node[tensorPhysical, minimum size=\smallTensorWidth] (NpI) at (\posXNpI, \posYNpI) {};
		\node[tensorPhysical, minimum size=\smallTensorWidth] (NpJ) at (\posXNpJ, \posYNpJ) {};
		\node[tensorPhysical, minimum size=\smallTensorWidth, fill=greenMarkingColor] (NpK) at (\posXNpK, \posYNpK) {};
		\node[tensorOrthoCenter, minimum size=\smallTensorWidth] (NpL) at (\posXNpL, \posYNpL) {};
		\node[tensorPhysical, minimum size=\smallTensorWidth, fill=greenMarkingColor] (NpM) at (\posXNpM, \posYNpM) {};
		
		\begin{pgfonlayer}{bg}
			% Draw subregion M
			%\fill[color=greenMarkingColor, opacity=\backgroundopacity] \convexhull{K, M, L}{\smallTensorWidth*0.9};
			%\node[color=greenMarkingColor] (Mp) at (\posXNpK-\tensorDistance*0.2, \posYNpK-\tensorDistance*0.4) {$\mathcal{M}^\prime$};
			
			% Draw connecting legs
			\draw[virtualLeg] (NpA) -- (NpB);
			\draw[virtualLeg] (NpB) -- (NpF);
			\draw[virtualLeg] (NpC) -- (NpB);
			\draw[virtualLeg] (NpC) -- (NpK);
			\draw[virtualLeg] (NpF) -- (NpG);
			\draw[virtualLeg] (NpF) -- (NpM);
			\draw[virtualLeg] (NpG) -- (NpH);
			\draw[virtualLeg] (NpH) -- (NpL);
			\draw[virtualLeg] (NpI) -- (NpJ);
			\draw[virtualLeg] (NpI) -- (NpG);
			\draw[virtualLeg] (NpJ) -- (NpG);
			\draw[virtualLeg] (NpK) -- (NpL);
			\draw[virtualLeg] (NpK) -- (NpM);
			\draw[virtualLeg] (NpM) -- (NpL);
			
			\draw[virtualLeg] ({\posXNpA+\openLegLength*cos(\angleA)}, {\posYNpA+\openLegLength*sin(\angleA)}) -- (NpA);
			\draw[virtualLeg] ({\posXNpC+\openLegLength*cos(\angleC)}, {\posYNpC+\openLegLength*sin(\angleC)}) -- (NpC);
			\draw[virtualLeg] ({\posXNpF+\openLegLength*cos(\angleF)}, {\posYNpF+\openLegLength*sin(\angleF)}) -- (NpF);
			\draw[virtualLeg] ({\posXNpH+\openLegLength*cos(\angleH)}, {\posYNpH+\openLegLength*sin(\angleH)}) -- (NpH);
			\draw[virtualLeg] ({\posXNpI+\openLegLength*cos(\angleI)}, {\posYNpI+\openLegLength*sin(\angleI)}) -- (NpI);
			\draw[virtualLeg] ({\posXNpJ+\openLegLength*cos(\angleJ)}, {\posYNpJ+\openLegLength*sin(\angleJ)}) -- (NpJ);
		\end{pgfonlayer}
		
		% Draw connections of upper/lower wavefunction
		\begin{pgfonlayer}{bg}
			\draw[virtualLegWithoutArrows] (NpA) -- ++({\openLegLength*cos(\angleA)}, {\openLegLength*sin(\angleA)}) -- ++(0, -\ccdistance) -- (NA);
			\draw[virtualLegWithoutArrows] (NpF) -- ++({\openLegLength*cos(\angleF)}, {\openLegLength*sin(\angleF)}) -- ++(0, -\ccdistance) -- (NF);
			\draw[virtualLegWithoutArrows] (NpH) -- ++({\openLegLength*cos(\angleH)}, {\openLegLength*sin(\angleH)}) -- ++(0, -\ccdistance) -- (NH);
			\draw[virtualLegWithoutArrows] (NpI) -- ++({\openLegLength*cos(\angleI)}, {\openLegLength*sin(\angleI)}) -- ++(0, -\ccdistance) -- (NI);
			\draw[virtualLegWithoutArrows] (NpJ) -- ++({\openLegLength*cos(\angleJ)}, {\openLegLength*sin(\angleJ)}) -- ++(0, -\ccdistance) -- (NJ);
		\end{pgfonlayer}
		\draw[virtualLegWithoutArrows] (NpC) -- ++({\openLegLength*cos(\angleC)}, {\openLegLength*sin(\angleC)}) -- ++(0, -\ccdistance) -- (NC);
		
		% Draw second equal sign
		\node[] () at ({\posXNI+\openLegLength*cos(\angleI)+\defaultDistanceEquations/2}, 0) {$=$};
		
		\def\xoffset{158pt}
		
		\def\posXNA{0 + \xoffset}
		\def\posYNA{-\ccdistance/2}
		\def\posXNB{\tensorDistance*0.8 + \xoffset}
		\def\posYNB{\tensorDistance*0.3-\ccdistance/2}
		\def\posXNC{\tensorDistance*0.5 + \xoffset}
		\def\posYNC{-\tensorDistance*0.4-\ccdistance/2}
		\def\posXND{\tensorDistance*1.2 + \xoffset}
		\def\posYND{-\tensorDistance*0.5-\ccdistance/2}
		\def\posXNE{\tensorDistance*1.8 + \xoffset}
		\def\posYNE{-\tensorDistance*0.45-\ccdistance/2}
		\def\posXNF{\tensorDistance*1.5 + \xoffset}
		\def\posYNF{\tensorDistance*0.4-\ccdistance/2}
		\def\posXNG{\tensorDistance*2.3 + \xoffset}
		\def\posYNG{-\tensorDistance*0.1-\ccdistance/2}
		\def\posXNH{\tensorDistance*2.2 + \xoffset}
		\def\posYNH{-\tensorDistance*0.8-\ccdistance/2}
		\def\posXNI{\tensorDistance*2.9 + \xoffset}
		\def\posYNI{-\tensorDistance*0.3-\ccdistance/2}
		\def\posXNJ{\tensorDistance*3.1 + \xoffset}
		\def\posYNJ{\tensorDistance*0.4-\ccdistance/2}
		
		\def\posXNpA{0}
		\def\posYNpA{\ccdistance/2}
		\def\posXNpB{\tensorDistance*0.8 + \xoffset}
		\def\posYNpB{\tensorDistance*0.3+\ccdistance/2}
		\def\posXNpC{\tensorDistance*0.5 + \xoffset}
		\def\posYNpC{-\tensorDistance*0.4+\ccdistance/2}
		\def\posXNpF{\tensorDistance*1.5 + \xoffset}
		\def\posYNpF{\tensorDistance*0.4+\ccdistance/2}
		\def\posXNpG{\tensorDistance*2.3 + \xoffset}
		\def\posYNpG{-\tensorDistance*0.1+\ccdistance/2}
		\def\posXNpH{\tensorDistance*2.2 + \xoffset}
		\def\posYNpH{-\tensorDistance*0.8+\ccdistance/2}
		\def\posXNpI{\tensorDistance*2.9 + \xoffset}
		\def\posYNpI{-\tensorDistance*0.3+\ccdistance/2}
		\def\posXNpJ{\tensorDistance*3.1 + \xoffset}
		\def\posYNpJ{\tensorDistance*0.4+\ccdistance/2}
		\def\posXNpK{\tensorDistance*1.1 + \xoffset}
		\def\posYNpK{-\tensorDistance*0.55+\ccdistance/2}
		\def\posXNpL{\tensorDistance*1.7 + \xoffset}
		\def\posYNpL{-\tensorDistance*0.6+\ccdistance/2}
		\def\posXNpM{\tensorDistance*1.45 + \xoffset}
		\def\posYNpM{-\tensorDistance*0.15+\ccdistance/2}
		
		% Draw tensors of N
		\node[tensorPhysical, minimum size=\smallTensorWidth] (NB) at (\posXNB, \posYNB) {};
		\node[tensorOrthoCenter, minimum size=\smallTensorWidth] (ND) at (\posXND, \posYND) {};
		\node[tensorPhysical, minimum size=\smallTensorWidth, fill=redMarkingColor] (NE) at (\posXNE, \posYNE) {};
		\node[tensorPhysical, minimum size=\smallTensorWidth] (NF) at (\posXNF, \posYNF) {};
		\node[tensorPhysical, minimum size=\smallTensorWidth] (NG) at (\posXNG, \posYNG) {};
		\node[tensorPhysical, minimum size=\smallTensorWidth] (NH) at (\posXNH, \posYNH) {};
		\node[tensorPhysical, minimum size=\smallTensorWidth] (NJ) at (\posXNJ, \posYNJ) {};
		
		\def\angleBa{(-180)}
		\def\angleBb{(-140)}
		\def\angleD{(180-atan(1/7))}
		\def\angleG{(-atan(1/8))}
		\def\angleJb{(-50)}
		\def\angleK{(180-atan(15/60))}
		
		\begin{pgfonlayer}{bg}
			% Draw subregion M
			%\fill[fill=redMarkingColor, opacity=\backgroundopacity] \convexhull{D, E}{\smallTensorWidth*0.9};
			%\node[color=redMarkingColor] (M) at (\posXND-\tensorDistance*0.2, \posYND-\tensorDistance*0.4) {$\mathcal{M}$};
			
			% Draw connecting legs
			\draw[virtualLeg] (NB) -- (NF);
			\draw[virtualLeg] (NE) -- (ND);
			\draw[virtualLeg] (NF) -- (NG);
			\draw[virtualLeg] (NF) -- (NE);
			\draw[virtualLeg] (NG) -- (NH);
			\draw[virtualLeg] (NH) -- (NE);
			\draw[virtualLeg] (NJ) -- (NG);
			
			\draw[virtualLeg] ({\posXNF+\openLegLength*cos(\angleF)}, {\posYNF+\openLegLength*sin(\angleF)}) -- (NF);
			\draw[virtualLeg] ({\posXNH+\openLegLength*cos(\angleH)}, {\posYNH+\openLegLength*sin(\angleH)}) -- (NH);
			\draw[virtualLeg] ({\posXNJ+\openLegLength*cos(\angleJ)}, {\posYNJ+\openLegLength*sin(\angleJ)}) -- (NJ);
			
			\draw[virtualLeg] ({\posXNB+\openLegLength*cos(\angleBa)}, {\posYNB+\openLegLength*sin(\angleBa)}) -- (NB);
			\draw[virtualLeg] ({\posXNB+\openLegLength*cos(\angleBb)}, {\posYNB+\openLegLength*sin(\angleBb)}) -- (NB);
			%\draw[virtualLeg] ({\posXND+\openLegLength*cos(\angleD)}, {\posYND+\openLegLength*sin(\angleD)}) -- (ND); 
			\draw[virtualLeg] ({\posXNpK+\openLegLength*cos(\angleK)}, {\posYNpK+\openLegLength*sin(\angleK)-\ccdistance}) -- (ND); % Draw now with different angle for better visual clarity!
			\draw[virtualLeg] ({\posXNG+\openLegLength*cos(\angleG)}, {\posYNG+\openLegLength*sin(\angleG)}) -- (NG);
			\draw[virtualLeg] ({\posXNJ+\openLegLength*cos(\angleJb)}, {\posYNJ+\openLegLength*sin(\angleJb)}) -- (NJ);
		\end{pgfonlayer}
		
		% Draw tensors of N'
		\node[tensorPhysical, minimum size=\smallTensorWidth] (NpB) at (\posXNpB, \posYNpB) {};
		\node[tensorPhysical, minimum size=\smallTensorWidth] (NpF) at (\posXNpF, \posYNpF) {};
		\node[tensorPhysical, minimum size=\smallTensorWidth] (NpG) at (\posXNpG, \posYNpG) {};
		\node[tensorPhysical, minimum size=\smallTensorWidth] (NpH) at (\posXNpH, \posYNpH) {};
		\node[tensorPhysical, minimum size=\smallTensorWidth] (NpJ) at (\posXNpJ, \posYNpJ) {};
		\node[tensorPhysical, minimum size=\smallTensorWidth, fill=greenMarkingColor] (NpK) at (\posXNpK, \posYNpK) {};
		\node[tensorOrthoCenter, minimum size=\smallTensorWidth] (NpL) at (\posXNpL, \posYNpL) {};
		\node[tensorPhysical, minimum size=\smallTensorWidth, fill=greenMarkingColor] (NpM) at (\posXNpM, \posYNpM) {};
		
		\begin{pgfonlayer}{bg}
			% Draw subregion M
			%\fill[color=greenMarkingColor, opacity=\backgroundopacity] \convexhull{K, M, L}{\smallTensorWidth*0.9};
			%\node[color=greenMarkingColor] (Mp) at (\posXNpK-\tensorDistance*0.2, \posYNpK-\tensorDistance*0.4) {$\mathcal{M}^\prime$};
			
			% Draw connecting legs
			\draw[virtualLeg] (NpB) -- (NpF);
			\draw[virtualLeg] (NpF) -- (NpG);
			\draw[virtualLeg] (NpF) -- (NpM);
			\draw[virtualLeg] (NpG) -- (NpH);
			\draw[virtualLeg] (NpH) -- (NpL);
			\draw[virtualLeg] (NpJ) -- (NpG);
			\draw[virtualLeg] (NpK) -- (NpL);
			\draw[virtualLeg] (NpK) -- (NpM);
			\draw[virtualLeg] (NpM) -- (NpL);
			
			\draw[virtualLeg] ({\posXNpF+\openLegLength*cos(\angleF)}, {\posYNpF+\openLegLength*sin(\angleF)}) -- (NpF);
			\draw[virtualLeg] ({\posXNpH+\openLegLength*cos(\angleH)}, {\posYNpH+\openLegLength*sin(\angleH)}) -- (NpH);
			\draw[virtualLeg] ({\posXNpJ+\openLegLength*cos(\angleJ)}, {\posYNpJ+\openLegLength*sin(\angleJ)}) -- (NpJ);
			
			\draw[virtualLeg] ({\posXNpB+\openLegLength*cos(\angleBa)}, {\posYNpB+\openLegLength*sin(\angleBa)}) -- (NpB);
			\draw[virtualLeg] ({\posXNpB+\openLegLength*cos(\angleBb)}, {\posYNpB+\openLegLength*sin(\angleBb)}) -- (NpB);
			\draw[virtualLeg] ({\posXNpK+\openLegLength*cos(\angleK)}, {\posYNpK+\openLegLength*sin(\angleK)}) -- (NpK);
			\draw[virtualLeg] ({\posXNpG+\openLegLength*cos(\angleG)}, {\posYNpG+\openLegLength*sin(\angleG)}) -- (NpG);
			\draw[virtualLeg] ({\posXNpJ+\openLegLength*cos(\angleJb)}, {\posYNpJ+\openLegLength*sin(\angleJb)}) -- (NpJ);
		\end{pgfonlayer}
		
		% Draw connections of upper/lower wavefunction
		\begin{pgfonlayer}{bg}
			\draw[virtualLegWithoutArrows] (NpF) -- ++({\openLegLength*cos(\angleF)}, {\openLegLength*sin(\angleF)}) -- ++(0, -\ccdistance) -- (NF);
			\draw[virtualLegWithoutArrows] (NpH) -- ++({\openLegLength*cos(\angleH)}, {\openLegLength*sin(\angleH)}) -- ++(0, -\ccdistance) -- (NH);
			\draw[virtualLegWithoutArrows] (NpJ) -- ++({\openLegLength*cos(\angleJ)}, {\openLegLength*sin(\angleJ)}) -- ++(0, -\ccdistance) -- (NJ);
			
			\draw[virtualLegWithoutArrows] (NpB) -- ++({\openLegLength*cos(\angleBa)}, {\openLegLength*sin(\angleBa)}) -- ++(0, -\ccdistance) -- (NB);
			\draw[virtualLegWithoutArrows] (NpB) -- ++({\openLegLength*cos(\angleBb)}, {\openLegLength*sin(\angleBb)}) -- ++(0, -\ccdistance) -- (NB);
			\draw[virtualLegWithoutArrows] (NpK) -- ++({\openLegLength*cos(\angleK)}, {\openLegLength*sin(\angleK)}) -- ++(0, -\ccdistance) -- (ND);
			
			\draw[virtualLegWithoutArrows] (NpG) -- ++({\openLegLength*cos(\angleG)}, {\openLegLength*sin(\angleG)})  -- ++(0, -\ccdistance)-- (NG);
			\draw[virtualLegWithoutArrows] (NpJ) -- ++({\openLegLength*cos(\angleJb)}, {\openLegLength*sin(\angleJb)})  -- ++(0, -\ccdistance)-- (NJ);
		\end{pgfonlayer}
		
		% Draw third equal sign
		\def\yoffset{-140pt}
		\def\xoffset{70pt}
		\node[] () at (\xoffset, \yoffset) {$=$};
		\node[] () at (\xoffset + \defaultDistanceEquations/2, \yoffset) {$\dots$};
		\node[] () at (\xoffset + \defaultDistanceEquations, \yoffset) {$=$};
		
		\def\additionalxoffset{25 pt}
		\def\additionalyoffset{25 pt}
		\def\ccdistance{50 pt}
		
		\def\posXND{\tensorDistance*1.2 + \xoffset + \additionalxoffset}
		\def\posYND{-\tensorDistance*0.5-\ccdistance/2 + \yoffset + \additionalyoffset}
		\def\posXNE{\tensorDistance*1.8 + \xoffset + \additionalxoffset}
		\def\posYNE{-\tensorDistance*0.45-\ccdistance/2 + \yoffset + \additionalyoffset}
		
		\def\posXNpK{\tensorDistance*1.2 + \xoffset + \additionalxoffset}
		\def\posYNpK{-\tensorDistance*0.55+\ccdistance/2 + \yoffset + \additionalyoffset}
		\def\posXNpL{\tensorDistance*1.8 + \xoffset + \additionalxoffset}
		\def\posYNpL{-\tensorDistance*0.6+\ccdistance/2 + \yoffset + \additionalyoffset}
		\def\posXNpM{\tensorDistance*1.55 + \xoffset + \additionalxoffset}
		\def\posYNpM{-\tensorDistance*0.15+\ccdistance/2 + \yoffset + \additionalyoffset}
		
		% Draw tensors of N
		\node[tensorOrthoCenter, minimum size=\smallTensorWidth] (ND) at (\posXND, \posYND) {};
		\node[tensorPhysical, minimum size=\smallTensorWidth, fill=redMarkingColor] (NE) at (\posXNE, \posYNE) {};
		
		\draw[virtualLeg] (NE) -- (ND);
		
		% Draw tensors of N'
		\node[tensorPhysical, minimum size=\smallTensorWidth, fill=greenMarkingColor] (NpK) at (\posXNpK, \posYNpK) {};
		\node[tensorOrthoCenter, minimum size=\smallTensorWidth] (NpL) at (\posXNpL, \posYNpL) {};
		\node[tensorPhysical, minimum size=\smallTensorWidth, fill=greenMarkingColor] (NpM) at (\posXNpM, \posYNpM) {};
		
		\draw[virtualLeg] (NpK) -- (NpL);
		\draw[virtualLeg] (NpM) -- (NpL);
		\draw[virtualLeg] (NpK) -- (NpM);
		
		\def\angleEa{60}
		\def\angleEb{-30}
		\def\angleD{-140}
		
		\def\angleK{-140}
		\def\angleL{-20}
		\def\angleM{30}
		
		\draw[virtualLeg] ({\posXNE+\openLegLength*cos(\angleEa)}, {\posYNE+\openLegLength*sin(\angleEa)}) -- (NE);
		\draw[virtualLeg] ({\posXNE+\openLegLength*cos(\angleEb)}, {\posYNE+\openLegLength*sin(\angleEb)}) -- (NE);
		\draw[virtualLeg] ({\posXND+\openLegLength*cos(\angleD)}, {\posYND+\openLegLength*sin(\angleD)}) -- (ND);
		\draw[virtualLeg] ({\posXNpK+\openLegLength*cos(\angleK)}, {\posYNpK+\openLegLength*sin(\angleK)}) -- (NpK);
		\draw[virtualLeg] ({\posXNpL+\openLegLength*cos(\angleL)}, {\posYNpL+\openLegLength*sin(\angleL)}) -- (NpL);
		\draw[virtualLeg] ({\posXNpM+\openLegLength*cos(\angleM)}, {\posYNpM+\openLegLength*sin(\angleM)}) -- (NpM);
		
		\draw[virtualLegWithoutArrows] (NpK) -- ++({\openLegLength*cos(\angleK)}, {\openLegLength*sin(\angleK)}) -- ({\posXND+\openLegLength*cos(\angleD)}, {\posYND+\openLegLength*sin(\angleD)}) -- (ND);
		\draw[virtualLegWithoutArrows] (NpM) -- ++({\openLegLength*cos(\angleM)}, {\openLegLength*sin(\angleM)}) -- ({\posXNE+\openLegLength*cos(\angleEa)}, {\posYNE+\openLegLength*sin(\angleEa)}) -- (NE);
		\draw[virtualLegWithoutArrows] (NpL) -- ++({\openLegLength*cos(\angleL)}, {\openLegLength*sin(\angleL)}) -- ({\posXNE+\openLegLength*cos(\angleEb)}, {\posYNE+\openLegLength*sin(\angleEb)}) -- (NE);
		
	\end{tikzpicture}
\end{document}